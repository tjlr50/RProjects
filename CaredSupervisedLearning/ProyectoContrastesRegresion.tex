% Options for packages loaded elsewhere
\PassOptionsToPackage{unicode}{hyperref}
\PassOptionsToPackage{hyphens}{url}
%
\documentclass[
]{article}
\usepackage{lmodern}
\usepackage{amssymb,amsmath}
\usepackage{ifxetex,ifluatex}
\ifnum 0\ifxetex 1\fi\ifluatex 1\fi=0 % if pdftex
  \usepackage[T1]{fontenc}
  \usepackage[utf8]{inputenc}
  \usepackage{textcomp} % provide euro and other symbols
\else % if luatex or xetex
  \usepackage{unicode-math}
  \defaultfontfeatures{Scale=MatchLowercase}
  \defaultfontfeatures[\rmfamily]{Ligatures=TeX,Scale=1}
\fi
% Use upquote if available, for straight quotes in verbatim environments
\IfFileExists{upquote.sty}{\usepackage{upquote}}{}
\IfFileExists{microtype.sty}{% use microtype if available
  \usepackage[]{microtype}
  \UseMicrotypeSet[protrusion]{basicmath} % disable protrusion for tt fonts
}{}
\makeatletter
\@ifundefined{KOMAClassName}{% if non-KOMA class
  \IfFileExists{parskip.sty}{%
    \usepackage{parskip}
  }{% else
    \setlength{\parindent}{0pt}
    \setlength{\parskip}{6pt plus 2pt minus 1pt}}
}{% if KOMA class
  \KOMAoptions{parskip=half}}
\makeatother
\usepackage{xcolor}
\IfFileExists{xurl.sty}{\usepackage{xurl}}{} % add URL line breaks if available
\IfFileExists{bookmark.sty}{\usepackage{bookmark}}{\usepackage{hyperref}}
\hypersetup{
  pdftitle={Proyecto - Aprendizaje supervisado con caret},
  pdfauthor={Tomas Lemus},
  hidelinks,
  pdfcreator={LaTeX via pandoc}}
\urlstyle{same} % disable monospaced font for URLs
\usepackage[margin=1in]{geometry}
\usepackage{color}
\usepackage{fancyvrb}
\newcommand{\VerbBar}{|}
\newcommand{\VERB}{\Verb[commandchars=\\\{\}]}
\DefineVerbatimEnvironment{Highlighting}{Verbatim}{commandchars=\\\{\}}
% Add ',fontsize=\small' for more characters per line
\usepackage{framed}
\definecolor{shadecolor}{RGB}{248,248,248}
\newenvironment{Shaded}{\begin{snugshade}}{\end{snugshade}}
\newcommand{\AlertTok}[1]{\textcolor[rgb]{0.94,0.16,0.16}{#1}}
\newcommand{\AnnotationTok}[1]{\textcolor[rgb]{0.56,0.35,0.01}{\textbf{\textit{#1}}}}
\newcommand{\AttributeTok}[1]{\textcolor[rgb]{0.77,0.63,0.00}{#1}}
\newcommand{\BaseNTok}[1]{\textcolor[rgb]{0.00,0.00,0.81}{#1}}
\newcommand{\BuiltInTok}[1]{#1}
\newcommand{\CharTok}[1]{\textcolor[rgb]{0.31,0.60,0.02}{#1}}
\newcommand{\CommentTok}[1]{\textcolor[rgb]{0.56,0.35,0.01}{\textit{#1}}}
\newcommand{\CommentVarTok}[1]{\textcolor[rgb]{0.56,0.35,0.01}{\textbf{\textit{#1}}}}
\newcommand{\ConstantTok}[1]{\textcolor[rgb]{0.00,0.00,0.00}{#1}}
\newcommand{\ControlFlowTok}[1]{\textcolor[rgb]{0.13,0.29,0.53}{\textbf{#1}}}
\newcommand{\DataTypeTok}[1]{\textcolor[rgb]{0.13,0.29,0.53}{#1}}
\newcommand{\DecValTok}[1]{\textcolor[rgb]{0.00,0.00,0.81}{#1}}
\newcommand{\DocumentationTok}[1]{\textcolor[rgb]{0.56,0.35,0.01}{\textbf{\textit{#1}}}}
\newcommand{\ErrorTok}[1]{\textcolor[rgb]{0.64,0.00,0.00}{\textbf{#1}}}
\newcommand{\ExtensionTok}[1]{#1}
\newcommand{\FloatTok}[1]{\textcolor[rgb]{0.00,0.00,0.81}{#1}}
\newcommand{\FunctionTok}[1]{\textcolor[rgb]{0.00,0.00,0.00}{#1}}
\newcommand{\ImportTok}[1]{#1}
\newcommand{\InformationTok}[1]{\textcolor[rgb]{0.56,0.35,0.01}{\textbf{\textit{#1}}}}
\newcommand{\KeywordTok}[1]{\textcolor[rgb]{0.13,0.29,0.53}{\textbf{#1}}}
\newcommand{\NormalTok}[1]{#1}
\newcommand{\OperatorTok}[1]{\textcolor[rgb]{0.81,0.36,0.00}{\textbf{#1}}}
\newcommand{\OtherTok}[1]{\textcolor[rgb]{0.56,0.35,0.01}{#1}}
\newcommand{\PreprocessorTok}[1]{\textcolor[rgb]{0.56,0.35,0.01}{\textit{#1}}}
\newcommand{\RegionMarkerTok}[1]{#1}
\newcommand{\SpecialCharTok}[1]{\textcolor[rgb]{0.00,0.00,0.00}{#1}}
\newcommand{\SpecialStringTok}[1]{\textcolor[rgb]{0.31,0.60,0.02}{#1}}
\newcommand{\StringTok}[1]{\textcolor[rgb]{0.31,0.60,0.02}{#1}}
\newcommand{\VariableTok}[1]{\textcolor[rgb]{0.00,0.00,0.00}{#1}}
\newcommand{\VerbatimStringTok}[1]{\textcolor[rgb]{0.31,0.60,0.02}{#1}}
\newcommand{\WarningTok}[1]{\textcolor[rgb]{0.56,0.35,0.01}{\textbf{\textit{#1}}}}
\usepackage{graphicx}
\makeatletter
\def\maxwidth{\ifdim\Gin@nat@width>\linewidth\linewidth\else\Gin@nat@width\fi}
\def\maxheight{\ifdim\Gin@nat@height>\textheight\textheight\else\Gin@nat@height\fi}
\makeatother
% Scale images if necessary, so that they will not overflow the page
% margins by default, and it is still possible to overwrite the defaults
% using explicit options in \includegraphics[width, height, ...]{}
\setkeys{Gin}{width=\maxwidth,height=\maxheight,keepaspectratio}
% Set default figure placement to htbp
\makeatletter
\def\fps@figure{htbp}
\makeatother
\setlength{\emergencystretch}{3em} % prevent overfull lines
\providecommand{\tightlist}{%
  \setlength{\itemsep}{0pt}\setlength{\parskip}{0pt}}
\setcounter{secnumdepth}{-\maxdimen} % remove section numbering
\usepackage{booktabs}
\usepackage{longtable}
\usepackage{array}
\usepackage{multirow}
\usepackage{wrapfig}
\usepackage{float}
\usepackage{colortbl}
\usepackage{pdflscape}
\usepackage{tabu}
\usepackage{threeparttable}
\usepackage{threeparttablex}
\usepackage[normalem]{ulem}
\usepackage{makecell}
\usepackage{xcolor}
\ifluatex
  \usepackage{selnolig}  % disable illegal ligatures
\fi

\title{Proyecto - Aprendizaje supervisado con caret}
\author{Tomas Lemus}
\date{20/11/2020}

\begin{document}
\maketitle

\begin{Shaded}
\begin{Highlighting}[]
\DocumentationTok{\#\#install.packages("readr")}
\CommentTok{\#install.packages("tidyverse")}
\end{Highlighting}
\end{Shaded}

\begin{Shaded}
\begin{Highlighting}[]
\NormalTok{df }\OtherTok{=} \FunctionTok{read\_csv}\NormalTok{(}\StringTok{"Alzheimer.csv"}\NormalTok{)}
\end{Highlighting}
\end{Shaded}

\begin{verbatim}
## 
## -- Column specification --------------------------------------------------------
## cols(
##   .default = col_double(),
##   SEX = col_character(),
##   CLASS = col_character()
## )
## i Use `spec()` for the full column specifications.
\end{verbatim}

\begin{Shaded}
\begin{Highlighting}[]
\FunctionTok{head}\NormalTok{(df)}
\end{Highlighting}
\end{Shaded}

\begin{verbatim}
## # A tibble: 6 x 228
##     AGE SEX   BRAIN_VOLUME GM_VOLUME WM_VOLUME CSF_VOLUME GM_BRAIN_QUOTIE~
##   <dbl> <chr>        <dbl>     <dbl>     <dbl>      <dbl>            <dbl>
## 1  68   MALE         1411.      599.      433.       379.            0.425
## 2  82.9 FEMA~        1368.      552.      446.       370.            0.403
## 3  72.8 FEMA~        1154.      444.      393.       317.            0.385
## 4  73   FEMA~        1547.      665.      492.       390.            0.430
## 5  71.1 FEMA~        1326.      559.      433.       334.            0.422
## 6  78.1 MALE         1400.      526.      441.       433.            0.376
## # ... with 221 more variables: WM_BRAIN_QUOTIENT <dbl>,
## #   CSF_BRAIN_QUOTIENT <dbl>, GM_WM_QUOTIENT <dbl>, PRECENTRAL_L_VOLUME <dbl>,
## #   PRECENTRAL_R_VOLUME <dbl>, FRONTAL_SUP_L_VOLUME <dbl>,
## #   FRONTAL_SUP_R_VOLUME <dbl>, FRONTAL_SUP_ORB_L_VOLUME <dbl>,
## #   FRONTAL_SUP_ORB_R_VOLUME <dbl>, FRONTAL_MID_L_VOLUME <dbl>,
## #   FRONTAL_MID_R_VOLUME <dbl>, FRONTAL_MID_ORB_L_VOLUME <dbl>,
## #   FRONTAL_MID_ORB_R_VOLUME <dbl>, FRONTAL_INF_OPER_L_VOLUME <dbl>,
## #   FRONTAL_INF_OPER_R_VOLUME <dbl>, FRONTAL_INF_TRI_L_VOLUME <dbl>,
## #   FRONTAL_INF_TRI_R_VOLUME <dbl>, FRONTAL_INF_ORB_L_VOLUME <dbl>,
## #   FRONTAL_INF_ORB_R_VOLUME <dbl>, ROLANDIC_OPER_L_VOLUME <dbl>,
## #   ROLANDIC_OPER_R_VOLUME <dbl>, SUPP_MOTOR_AREA_L_VOLUME <dbl>,
## #   SUPP_MOTOR_AREA_R_VOLUME <dbl>, OLFACTORY_L_VOLUME <dbl>,
## #   OLFACTORY_R_VOLUME <dbl>, FRONTAL_SUP_MEDIAL_L_VOLUME <dbl>,
## #   FRONTAL_SUP_MEDIAL_R_VOLUME <dbl>, RECTUS_L_VOLUME <dbl>,
## #   RECTUS_R_VOLUME <dbl>, INSULA_L_VOLUME <dbl>, INSULA_R_VOLUME <dbl>,
## #   CINGULUM_ANT_L_VOLUME <dbl>, CINGULUM_ANT_R_VOLUME <dbl>,
## #   CINGULUM_MID_L_VOLUME <dbl>, CINGULUM_MID_R_VOLUME <dbl>,
## #   CINGULUM_POST_L_VOLUME <dbl>, CINGULUM_POST_R_VOLUME <dbl>,
## #   HIPPOCAMPUS_L_VOLUME <dbl>, HIPPOCAMPUS_R_VOLUME <dbl>,
## #   PARAHIPPOCAMPAL_L_VOLUME <dbl>, PARAHIPPOCAMPAL_R_VOLUME <dbl>,
## #   AMYGDALA_L_VOLUME <dbl>, AMYGDALA_R_VOLUME <dbl>, CALCARINE_L_VOLUME <dbl>,
## #   CALCARINE_R_VOLUME <dbl>, CUNEUS_L_VOLUME <dbl>, CUNEUS_R_VOLUME <dbl>,
## #   LINGUAL_L_VOLUME <dbl>, LINGUAL_R_VOLUME <dbl>,
## #   OCCIPITAL_SUP_L_VOLUME <dbl>, OCCIPITAL_SUP_R_VOLUME <dbl>,
## #   OCCIPITAL_MID_L_VOLUME <dbl>, OCCIPITAL_MID_R_VOLUME <dbl>,
## #   OCCIPITAL_INF_L_VOLUME <dbl>, OCCIPITAL_INF_R_VOLUME <dbl>,
## #   FUSIFORM_L_VOLUME <dbl>, FUSIFORM_R_VOLUME <dbl>,
## #   POSTCENTRAL_L_VOLUME <dbl>, POSTCENTRAL_R_VOLUME <dbl>,
## #   PARIETAL_SUP_L_VOLUME <dbl>, PARIETAL_SUP_R_VOLUME <dbl>,
## #   PARIETAL_INF_L_VOLUME <dbl>, PARIETAL_INF_R_VOLUME <dbl>,
## #   SUPRAMARGINAL_L_VOLUME <dbl>, SUPRAMARGINAL_R_VOLUME <dbl>,
## #   ANGULAR_L_VOLUME <dbl>, ANGULAR_R_VOLUME <dbl>, PRECUNEUS_L_VOLUME <dbl>,
## #   PRECUNEUS_R_VOLUME <dbl>, PARACENTRAL_LOBULE_L_VOLUME <dbl>,
## #   PARACENTRAL_LOBULE_R_VOLUME <dbl>, CAUDATE_L_VOLUME <dbl>,
## #   CAUDATE_R_VOLUME <dbl>, PUTAMEN_L_VOLUME <dbl>, PUTAMEN_R_VOLUME <dbl>,
## #   PALLIDUM_L_VOLUME <dbl>, PALLIDUM_R_VOLUME <dbl>, THALAMUS_L_VOLUME <dbl>,
## #   THALAMUS_R_VOLUME <dbl>, HESCHL_L_VOLUME <dbl>, HESCHL_R_VOLUME <dbl>,
## #   TEMPORAL_SUP_L_VOLUME <dbl>, TEMPORAL_SUP_R_VOLUME <dbl>,
## #   TEMPORAL_POLE_SUP_L_VOLUME <dbl>, TEMPORAL_POLE_SUP_R_VOLUME <dbl>,
## #   TEMPORAL_MID_L_VOLUME <dbl>, TEMPORAL_MID_R_VOLUME <dbl>,
## #   TEMPORAL_POLE_MID_L_VOLUME <dbl>, TEMPORAL_POLE_MID_R_VOLUME <dbl>,
## #   TEMPORAL_INF_L_VOLUME <dbl>, TEMPORAL_INF_R_VOLUME <dbl>,
## #   CEREBELUM_CRUS1_L_VOLUME <dbl>, CEREBELUM_CRUS1_R_VOLUME <dbl>,
## #   CEREBELUM_CRUS2_L_VOLUME <dbl>, CEREBELUM_CRUS2_R_VOLUME <dbl>,
## #   CEREBELUM_3_L_VOLUME <dbl>, CEREBELUM_3_R_VOLUME <dbl>,
## #   CEREBELUM_4_5_L_VOLUME <dbl>, CEREBELUM_4_5_R_VOLUME <dbl>,
## #   CEREBELUM_6_L_VOLUME <dbl>, ...
\end{verbatim}

\begin{Shaded}
\begin{Highlighting}[]
\FunctionTok{print}\NormalTok{(}\FunctionTok{dim}\NormalTok{(df))}
\end{Highlighting}
\end{Shaded}

\begin{verbatim}
## [1] 262 228
\end{verbatim}

Brevemente podemos observar sujetos masculinos y femeninos, desde 55
años de edad con una media de 74, además de variables que representan
medidas volumétricas del cerebro (volumen y grosor de regiones
anatómicas cerebrales) para personas sanas o con Alzheimer determinadas
por la categoria CLASS.Este dataset contiene inicialmente 262 registros
con 228 variables entre los cuales seleccionaremos segun sea necesario.

\begin{Shaded}
\begin{Highlighting}[]
\FunctionTok{select}\NormalTok{(df,CLASS,BRAIN\_VOLUME,SEX)}
\end{Highlighting}
\end{Shaded}

\begin{verbatim}
## # A tibble: 262 x 3
##    CLASS BRAIN_VOLUME SEX   
##    <chr>        <dbl> <chr> 
##  1 AD           1411. MALE  
##  2 AD           1368. FEMALE
##  3 AD           1154. FEMALE
##  4 AD           1547. FEMALE
##  5 AD           1326. FEMALE
##  6 AD           1400. MALE  
##  7 AD           1241. FEMALE
##  8 AD           1514. FEMALE
##  9 AD           1238. FEMALE
## 10 AD           1513. MALE  
## # ... with 252 more rows
\end{verbatim}

\begin{Shaded}
\begin{Highlighting}[]
\NormalTok{df}\OtherTok{=}\FunctionTok{select}\NormalTok{(df,CLASS,BRAIN\_VOLUME,SEX)}
\NormalTok{df\_H}\OtherTok{=}\NormalTok{df}\SpecialCharTok{\%\textgreater{}\%} \FunctionTok{filter}\NormalTok{(CLASS}\SpecialCharTok{==}\StringTok{"HEALTHY"}\NormalTok{,SEX}\SpecialCharTok{==}\StringTok{"MALE"}\NormalTok{)}
\NormalTok{df\_A}\OtherTok{=}\NormalTok{df}\SpecialCharTok{\%\textgreater{}\%} \FunctionTok{filter}\NormalTok{(CLASS}\SpecialCharTok{==}\StringTok{"AD"}\NormalTok{,SEX}\SpecialCharTok{==}\StringTok{"MALE"}\NormalTok{)}
\FunctionTok{head}\NormalTok{(df\_H)}
\end{Highlighting}
\end{Shaded}

\begin{verbatim}
## # A tibble: 6 x 3
##   CLASS   BRAIN_VOLUME SEX  
##   <chr>          <dbl> <chr>
## 1 HEALTHY        1608. MALE 
## 2 HEALTHY        1349. MALE 
## 3 HEALTHY        1391. MALE 
## 4 HEALTHY        1519. MALE 
## 5 HEALTHY        1602. MALE 
## 6 HEALTHY        1457. MALE
\end{verbatim}

CONTRASTE DE HIPÓTESIS: Se supone normalidad y homocedasticidad u
homogeneidad de varianzas, podemos realizar nuestro contraste. ha:
μAD\textless μH la variable volumen cerebral es significativamente menor
en sujetos con Alzheimer que en sujetos sanos.(\textless) ho: μAD
\textgreater= μH volumen cerebral de AD no es menor que en sanos

\begin{Shaded}
\begin{Highlighting}[]
\FunctionTok{t.test}\NormalTok{(df\_A}\SpecialCharTok{$}\NormalTok{BRAIN\_VOLUME,df\_H}\SpecialCharTok{$}\NormalTok{BRAIN\_VOLUME, }\AttributeTok{alternative =} \StringTok{"less"}\NormalTok{)}
\end{Highlighting}
\end{Shaded}

\begin{verbatim}
## 
##  Welch Two Sample t-test
## 
## data:  df_A$BRAIN_VOLUME and df_H$BRAIN_VOLUME
## t = 0.023782, df = 14.19, p-value = 0.5093
## alternative hypothesis: true difference in means is less than 0
## 95 percent confidence interval:
##      -Inf 74.43254
## sample estimates:
## mean of x mean of y 
##  1542.332  1541.340
\end{verbatim}

No se encontró evidencia suficiente para descartar H0, por lo tanto se
estima que el volumen cerebral de los sujetos AD no es menos que
aquellos sanos de sexo masculino.

\begin{Shaded}
\begin{Highlighting}[]
\FunctionTok{boxplot}\NormalTok{(df\_A}\SpecialCharTok{$}\NormalTok{BRAIN\_VOLUME,df\_H}\SpecialCharTok{$}\NormalTok{BRAIN\_VOLUME)}
\end{Highlighting}
\end{Shaded}

\includegraphics{ProyectoContrastesRegresion_files/figure-latex/unnamed-chunk-8-1.pdf}
Proceso para sexo femenino:

\begin{Shaded}
\begin{Highlighting}[]
\NormalTok{dFH}\OtherTok{=}\NormalTok{df}\SpecialCharTok{\%\textgreater{}\%} \FunctionTok{filter}\NormalTok{(CLASS}\SpecialCharTok{==}\StringTok{"HEALTHY"}\NormalTok{,SEX}\SpecialCharTok{==}\StringTok{"FEMALE"}\NormalTok{)}
\NormalTok{dFA}\OtherTok{=}\NormalTok{df}\SpecialCharTok{\%\textgreater{}\%} \FunctionTok{filter}\NormalTok{(CLASS}\SpecialCharTok{==}\StringTok{"AD"}\NormalTok{,SEX}\SpecialCharTok{==}\StringTok{"FEMALE"}\NormalTok{)}
\end{Highlighting}
\end{Shaded}

ha: μAD\textless μH la variable volumen cerebral es significativamente
menor en sujetos con Alzheimer que en sujetos sanos de sexo
femenino.(\textless) ho: μAD \textgreater= μH volumen cerebral de AD no
es menor que en sanos de sexo feminino.

\begin{Shaded}
\begin{Highlighting}[]
\FunctionTok{t.test}\NormalTok{(dFA}\SpecialCharTok{$}\NormalTok{BRAIN\_VOLUME,dFH}\SpecialCharTok{$}\NormalTok{BRAIN\_VOLUME, }\AttributeTok{alternative =} \StringTok{"less"}\NormalTok{)}
\end{Highlighting}
\end{Shaded}

\begin{verbatim}
## 
##  Welch Two Sample t-test
## 
## data:  dFA$BRAIN_VOLUME and dFH$BRAIN_VOLUME
## t = -0.64776, df = 19.635, p-value = 0.2623
## alternative hypothesis: true difference in means is less than 0
## 95 percent confidence interval:
##      -Inf 28.21545
## sample estimates:
## mean of x mean of y 
##  1322.434  1339.380
\end{verbatim}

No se encontró evidencia suficiente para descartar H0, por lo tanto se
estima que el volumen cerebral de los sujetos AD no es menos que
aquellos sanos de sexo femenino.

\begin{Shaded}
\begin{Highlighting}[]
\FunctionTok{boxplot}\NormalTok{(dFA}\SpecialCharTok{$}\NormalTok{BRAIN\_VOLUME,dFH}\SpecialCharTok{$}\NormalTok{BRAIN\_VOLUME)}
\end{Highlighting}
\end{Shaded}

\includegraphics{ProyectoContrastesRegresion_files/figure-latex/unnamed-chunk-11-1.pdf}

Repetir el análisis anterior, pero para comprobar si las variables de
volumen de sustancia gris o de sustancia blanca son diferentes entre
sujetos sanos y sujetos con AD (realizar un contraste por cada
variable).

\begin{Shaded}
\begin{Highlighting}[]
\CommentTok{\#masculinos}
\NormalTok{df2 }\OtherTok{=} \FunctionTok{read\_csv}\NormalTok{(}\StringTok{"Alzheimer.csv"}\NormalTok{)}
\end{Highlighting}
\end{Shaded}

\begin{verbatim}
## 
## -- Column specification --------------------------------------------------------
## cols(
##   .default = col_double(),
##   SEX = col_character(),
##   CLASS = col_character()
## )
## i Use `spec()` for the full column specifications.
\end{verbatim}

\begin{Shaded}
\begin{Highlighting}[]
\NormalTok{dfAM}\OtherTok{=}\NormalTok{ df2 }\SpecialCharTok{\%\textgreater{}\%} \FunctionTok{select}\NormalTok{(SEX,CLASS,GM\_VOLUME,WM\_VOLUME) }\SpecialCharTok{\%\textgreater{}\%} \FunctionTok{filter}\NormalTok{(SEX}\SpecialCharTok{==}\StringTok{"MALE"}\NormalTok{,CLASS}\SpecialCharTok{==}\StringTok{"AD"}\NormalTok{)}
\NormalTok{dfHM}\OtherTok{=}\NormalTok{ df2 }\SpecialCharTok{\%\textgreater{}\%} \FunctionTok{select}\NormalTok{(SEX,CLASS,GM\_VOLUME,WM\_VOLUME) }\SpecialCharTok{\%\textgreater{}\%} \FunctionTok{filter}\NormalTok{(SEX}\SpecialCharTok{==}\StringTok{"MALE"}\NormalTok{,CLASS}\SpecialCharTok{==}\StringTok{"HEALTHY"}\NormalTok{)}
\FunctionTok{head}\NormalTok{(dfAM)}
\end{Highlighting}
\end{Shaded}

\begin{verbatim}
## # A tibble: 6 x 4
##   SEX   CLASS GM_VOLUME WM_VOLUME
##   <chr> <chr>     <dbl>     <dbl>
## 1 MALE  AD         599.      433.
## 2 MALE  AD         526.      441.
## 3 MALE  AD         581.      515.
## 4 MALE  AD         602.      436.
## 5 MALE  AD         695.      497.
## 6 MALE  AD         644.      520.
\end{verbatim}

CONTRASTE DE HIPÓTESIS:

ha: μADG!=μADW las variables de volumen de sustancia gris o de sustancia
blanca son diferentes entre sujetos con AD ho: μADG = μADW las variables
de volumen de sustancia gris o de sustancia blanca no son diferentes
entre sujetos con AD

\begin{Shaded}
\begin{Highlighting}[]
\FunctionTok{t.test}\NormalTok{(dfAM}\SpecialCharTok{$}\NormalTok{GM\_VOLUME,dfAM}\SpecialCharTok{$}\NormalTok{WM\_VOLUME, }\AttributeTok{alternative =} \StringTok{"two.sided"}\NormalTok{)}
\end{Highlighting}
\end{Shaded}

\begin{verbatim}
## 
##  Welch Two Sample t-test
## 
## data:  dfAM$GM_VOLUME and dfAM$WM_VOLUME
## t = 6.213, df = 18.012, p-value = 7.285e-06
## alternative hypothesis: true difference in means is not equal to 0
## 95 percent confidence interval:
##  107.0281 216.3849
## sample estimates:
## mean of x mean of y 
##  649.8218  488.1153
\end{verbatim}

Valor p menor que 0.05 por lo tanto hay evidencia suficiente para
rechazar h0, y se estima diferencia entre las sustancias blancas y gris
para sujetos con AD masculinos.

\begin{Shaded}
\begin{Highlighting}[]
\FunctionTok{head}\NormalTok{(dfHM)}
\end{Highlighting}
\end{Shaded}

\begin{verbatim}
## # A tibble: 6 x 4
##   SEX   CLASS   GM_VOLUME WM_VOLUME
##   <chr> <chr>       <dbl>     <dbl>
## 1 MALE  HEALTHY      701.      494.
## 2 MALE  HEALTHY      558.      428.
## 3 MALE  HEALTHY      606.      409.
## 4 MALE  HEALTHY      667.      471.
## 5 MALE  HEALTHY      694.      521.
## 6 MALE  HEALTHY      626.      506.
\end{verbatim}

CONTRASTE DE HIPÓTESIS:

ha: μHG!=μHW las variables de volumen de sustancia gris o de sustancia
blanca son diferentes entre sujetos sanos. ho: μHG = μHW las variables
de volumen de sustancia gris o de sustancia blanca no son diferentes
entre sujetos sanos.

\begin{Shaded}
\begin{Highlighting}[]
\FunctionTok{t.test}\NormalTok{(dfHM}\SpecialCharTok{$}\NormalTok{GM\_VOLUME,dfHM}\SpecialCharTok{$}\NormalTok{WM\_VOLUME, }\AttributeTok{alternative =} \StringTok{"two.sided"}\NormalTok{)}
\end{Highlighting}
\end{Shaded}

\begin{verbatim}
## 
##  Welch Two Sample t-test
## 
## data:  dfHM$GM_VOLUME and dfHM$WM_VOLUME
## t = 20.271, df = 187.88, p-value < 2.2e-16
## alternative hypothesis: true difference in means is not equal to 0
## 95 percent confidence interval:
##  146.9420 178.6243
## sample estimates:
## mean of x mean of y 
##  664.1903  501.4072
\end{verbatim}

Valor p menor que 0.05 por lo tanto hay evidencia suficiente para
rechazar h0, y se estima diferencia entre las sustancias blancas y gris
para sujetos sanos de sexo masculino.

\hypertarget{analisis-para-sexo-femenino}{%
\subsection{Analisis para sexo
femenino}\label{analisis-para-sexo-femenino}}

\begin{Shaded}
\begin{Highlighting}[]
\NormalTok{dfAF}\OtherTok{=}\NormalTok{ df2 }\SpecialCharTok{\%\textgreater{}\%} \FunctionTok{select}\NormalTok{(SEX,CLASS,GM\_VOLUME,WM\_VOLUME) }\SpecialCharTok{\%\textgreater{}\%} \FunctionTok{filter}\NormalTok{(SEX}\SpecialCharTok{==}\StringTok{"FEMALE"}\NormalTok{,CLASS}\SpecialCharTok{==}\StringTok{"AD"}\NormalTok{)}
\NormalTok{dfHf}\OtherTok{=}\NormalTok{ df2 }\SpecialCharTok{\%\textgreater{}\%} \FunctionTok{select}\NormalTok{(SEX,CLASS,GM\_VOLUME,WM\_VOLUME) }\SpecialCharTok{\%\textgreater{}\%} \FunctionTok{filter}\NormalTok{(SEX}\SpecialCharTok{==}\StringTok{"FEMALE"}\NormalTok{,CLASS}\SpecialCharTok{==}\StringTok{"HEALTHY"}\NormalTok{)}
\FunctionTok{head}\NormalTok{(dfAF)}
\end{Highlighting}
\end{Shaded}

\begin{verbatim}
## # A tibble: 6 x 4
##   SEX    CLASS GM_VOLUME WM_VOLUME
##   <chr>  <chr>     <dbl>     <dbl>
## 1 FEMALE AD         552.      446.
## 2 FEMALE AD         444.      393.
## 3 FEMALE AD         665.      492.
## 4 FEMALE AD         559.      433.
## 5 FEMALE AD         409.      437.
## 6 FEMALE AD         615.      502.
\end{verbatim}

CONTRASTE DE HIPÓTESIS:

ha: μADG!=μADW las variables de volumen de sustancia gris o de sustancia
blanca son diferentes entre sujetos con AD de sexo femenino. ho: μADG =
μADW las variables de volumen de sustancia gris o de sustancia blanca no
son diferentes entre sujetos con AD de sexo femenino.

\begin{Shaded}
\begin{Highlighting}[]
\FunctionTok{t.test}\NormalTok{(dfAF}\SpecialCharTok{$}\NormalTok{GM\_VOLUME,dfAF}\SpecialCharTok{$}\NormalTok{WM\_VOLUME, }\AttributeTok{alternative =} \StringTok{"two.sided"}\NormalTok{)}
\end{Highlighting}
\end{Shaded}

\begin{verbatim}
## 
##  Welch Two Sample t-test
## 
## data:  dfAF$GM_VOLUME and dfAF$WM_VOLUME
## t = 7.4354, df = 23.426, p-value = 1.31e-07
## alternative hypothesis: true difference in means is not equal to 0
## 95 percent confidence interval:
##  102.7864 181.9164
## sample estimates:
## mean of x mean of y 
##  567.9276  425.5762
\end{verbatim}

Valor p menor que 0.05 por lo tanto hay evidencia suficiente para
rechazar h0, y se estima diferencia entre las sustancias blancas y gris
para sujetos con AD de sexo femenino.

\begin{Shaded}
\begin{Highlighting}[]
\FunctionTok{head}\NormalTok{(dfHf)}
\end{Highlighting}
\end{Shaded}

\begin{verbatim}
## # A tibble: 6 x 4
##   SEX    CLASS   GM_VOLUME WM_VOLUME
##   <chr>  <chr>       <dbl>     <dbl>
## 1 FEMALE HEALTHY      545.      365.
## 2 FEMALE HEALTHY      624.      414.
## 3 FEMALE HEALTHY      611.      467.
## 4 FEMALE HEALTHY      606.      471.
## 5 FEMALE HEALTHY      520.      372.
## 6 FEMALE HEALTHY      575.      438.
\end{verbatim}

CONTRASTE DE HIPÓTESIS:

ha: μHG!=μHW las variables de volumen de sustancia gris o de sustancia
blanca son diferentes entre sujetos sanos de sexo femenino. ho: μHG =
μHW las variables de volumen de sustancia gris o de sustancia blanca no
son diferentes entre sujetos sanos de sexo femenino.

\begin{Shaded}
\begin{Highlighting}[]
\FunctionTok{t.test}\NormalTok{(dfHf}\SpecialCharTok{$}\NormalTok{GM\_VOLUME,dfHf}\SpecialCharTok{$}\NormalTok{WM\_VOLUME, }\AttributeTok{alternative =} \StringTok{"two.sided"}\NormalTok{)}
\end{Highlighting}
\end{Shaded}

\begin{verbatim}
## 
##  Welch Two Sample t-test
## 
## data:  dfHf$GM_VOLUME and dfHf$WM_VOLUME
## t = 31.268, df = 231.26, p-value < 2.2e-16
## alternative hypothesis: true difference in means is not equal to 0
## 95 percent confidence interval:
##  160.9704 182.6209
## sample estimates:
## mean of x mean of y 
##  600.2010  428.4054
\end{verbatim}

Valor p menor que 0.05 por lo tanto hay evidencia suficiente para
rechazar h0, y se estima diferencia entre las sustancias blancas y gris
para sujetos sanos de sexo femenino.

¿Es diferente la incidencia de la enfermedad en pacientes de distinto
sexo?

¿Son independientes las variables SEX y CLASS? Usa un test χ2 para
comprobar la independencia y la correlación entre ambas.

H0: Son independientes las variables SEX y CLASS. H1: No son
independientes las variables SEX y CLASS.

\begin{Shaded}
\begin{Highlighting}[]
\NormalTok{tab }\OtherTok{\textless{}{-}} \FunctionTok{table}\NormalTok{(df}\SpecialCharTok{$}\NormalTok{SEX, df}\SpecialCharTok{$}\NormalTok{CLASS)}
\NormalTok{tab}
\end{Highlighting}
\end{Shaded}

\begin{verbatim}
##         
##           AD HEALTHY
##   FEMALE  17     137
##   MALE    13      95
\end{verbatim}

\begin{Shaded}
\begin{Highlighting}[]
\FunctionTok{chisq.test}\NormalTok{(tab)}
\end{Highlighting}
\end{Shaded}

\begin{verbatim}
## 
##  Pearson's Chi-squared test with Yates' continuity correction
## 
## data:  tab
## X-squared = 0.0027726, df = 1, p-value = 0.958
\end{verbatim}

Valor p mayor que 0.05 por lo tanto no hay evidencia suficiente para
rechazar h0, y se estima independencia entre las variables SEX y CLASS.

El índice χ2 toma el valor 0 cuando dos variables son independientes.

\hypertarget{regresion}{%
\section{REGRESION}\label{regresion}}

Nos planteamos establecer un patrón de atrofia anual (es decir,
dependiendo de la edad) en el cerebro. ¿Cómo varía el volumen del
cerebro con la edad?

Queremos establecer un modelo de regresión que use como predictores las
variables sexo y edad para estimar el valor del volumen cerebral
(variable BRAIN\_VOLUME).

\hypertarget{modelo-de-regresiuxf3n-simple-y-polinomico.}{%
\subsubsection{modelo de regresión simple y
polinomico.}\label{modelo-de-regresiuxf3n-simple-y-polinomico.}}

Hipotesis para modelos de regresión Ho: La función corresponde el modelo
H1: La función no corresponde al modelo Hipotesis para anova Ho: El
modelo más complejo no mejora al más simple H1: El modelo más complejo
mejora al más simple

\begin{Shaded}
\begin{Highlighting}[]
\NormalTok{dfR}\OtherTok{=}\NormalTok{ df2 }\SpecialCharTok{\%\textgreater{}\%} \FunctionTok{select}\NormalTok{(BRAIN\_VOLUME,AGE,SEX) }
\NormalTok{modelo\_1 }\OtherTok{\textless{}{-}} \FunctionTok{lm}\NormalTok{(BRAIN\_VOLUME }\SpecialCharTok{\textasciitilde{}}\NormalTok{ AGE, }\AttributeTok{data =}\NormalTok{ dfR)}
\NormalTok{modelo\_2 }\OtherTok{\textless{}{-}} \FunctionTok{lm}\NormalTok{(BRAIN\_VOLUME }\SpecialCharTok{\textasciitilde{}} \FunctionTok{poly}\NormalTok{(AGE, }\DecValTok{2}\NormalTok{), }\AttributeTok{data =}\NormalTok{ dfR)}
\NormalTok{modelo\_3 }\OtherTok{\textless{}{-}} \FunctionTok{lm}\NormalTok{(BRAIN\_VOLUME }\SpecialCharTok{\textasciitilde{}} \FunctionTok{poly}\NormalTok{(AGE, }\DecValTok{3}\NormalTok{), }\AttributeTok{data =}\NormalTok{ dfR)}
\NormalTok{modelo\_4 }\OtherTok{\textless{}{-}} \FunctionTok{lm}\NormalTok{(BRAIN\_VOLUME }\SpecialCharTok{\textasciitilde{}} \FunctionTok{poly}\NormalTok{(AGE, }\DecValTok{4}\NormalTok{), }\AttributeTok{data =}\NormalTok{ dfR)}
\NormalTok{modelo\_5 }\OtherTok{\textless{}{-}} \FunctionTok{lm}\NormalTok{(BRAIN\_VOLUME }\SpecialCharTok{\textasciitilde{}} \FunctionTok{poly}\NormalTok{(AGE, }\DecValTok{5}\NormalTok{), }\AttributeTok{data =}\NormalTok{ dfR)}

\FunctionTok{anova}\NormalTok{(modelo\_1, modelo\_2, modelo\_3, modelo\_4, modelo\_5)}
\end{Highlighting}
\end{Shaded}

\begin{verbatim}
## Analysis of Variance Table
## 
## Model 1: BRAIN_VOLUME ~ AGE
## Model 2: BRAIN_VOLUME ~ poly(AGE, 2)
## Model 3: BRAIN_VOLUME ~ poly(AGE, 3)
## Model 4: BRAIN_VOLUME ~ poly(AGE, 4)
## Model 5: BRAIN_VOLUME ~ poly(AGE, 5)
##   Res.Df     RSS Df Sum of Sq      F   Pr(>F)   
## 1    260 5547943                                
## 2    259 5381389  1    166554 8.0151 0.005007 **
## 3    258 5379546  1      1843 0.0887 0.766097   
## 4    257 5329182  1     50364 2.4237 0.120750   
## 5    256 5319691  1      9491 0.4568 0.499752   
## ---
## Signif. codes:  0 '***' 0.001 '**' 0.01 '*' 0.05 '.' 0.1 ' ' 1
\end{verbatim}

\begin{Shaded}
\begin{Highlighting}[]
\NormalTok{f1}\OtherTok{=} \FunctionTok{lm}\NormalTok{(dfR}\SpecialCharTok{$}\NormalTok{BRAIN\_VOLUME }\SpecialCharTok{\textasciitilde{}}\NormalTok{ dfR}\SpecialCharTok{$}\NormalTok{AGE }\SpecialCharTok{+}\NormalTok{ dfR}\SpecialCharTok{$}\NormalTok{SEX)}
\NormalTok{f2}\OtherTok{=} \FunctionTok{lm}\NormalTok{(dfR}\SpecialCharTok{$}\NormalTok{BRAIN\_VOLUME }\SpecialCharTok{\textasciitilde{}}\NormalTok{ dfR}\SpecialCharTok{$}\NormalTok{AGE }\SpecialCharTok{+}\NormalTok{ dfR}\SpecialCharTok{$}\NormalTok{SEX }\SpecialCharTok{+} \FunctionTok{I}\NormalTok{(dfR}\SpecialCharTok{$}\NormalTok{AGE}\SpecialCharTok{\^{}}\DecValTok{2}\NormalTok{))}
\NormalTok{f3}\OtherTok{=} \FunctionTok{lm}\NormalTok{(dfR}\SpecialCharTok{$}\NormalTok{BRAIN\_VOLUME }\SpecialCharTok{\textasciitilde{}}\NormalTok{ dfR}\SpecialCharTok{$}\NormalTok{AGE }\SpecialCharTok{+}\NormalTok{ dfR}\SpecialCharTok{$}\NormalTok{SEX }\SpecialCharTok{+} \FunctionTok{I}\NormalTok{(dfR}\SpecialCharTok{$}\NormalTok{AGE}\SpecialCharTok{\^{}}\DecValTok{2}\NormalTok{) }\SpecialCharTok{+} \FunctionTok{I}\NormalTok{(dfR}\SpecialCharTok{$}\NormalTok{AGE}\SpecialCharTok{\^{}}\DecValTok{3}\NormalTok{))}
\NormalTok{f4}\OtherTok{=} \FunctionTok{lm}\NormalTok{(dfR}\SpecialCharTok{$}\NormalTok{BRAIN\_VOLUME }\SpecialCharTok{\textasciitilde{}}\NormalTok{ dfR}\SpecialCharTok{$}\NormalTok{AGE }\SpecialCharTok{+}\NormalTok{ dfR}\SpecialCharTok{$}\NormalTok{SEX }\SpecialCharTok{+} \FunctionTok{I}\NormalTok{(dfR}\SpecialCharTok{$}\NormalTok{AGE}\SpecialCharTok{\^{}}\DecValTok{2}\NormalTok{) }\SpecialCharTok{+} \FunctionTok{I}\NormalTok{(dfR}\SpecialCharTok{$}\NormalTok{AGE}\SpecialCharTok{\^{}}\DecValTok{3}\NormalTok{) }\SpecialCharTok{+} \FunctionTok{I}\NormalTok{(dfR}\SpecialCharTok{$}\NormalTok{AGE}\SpecialCharTok{\^{}}\DecValTok{4}\NormalTok{))}
\NormalTok{f5}\OtherTok{=} \FunctionTok{lm}\NormalTok{(dfR}\SpecialCharTok{$}\NormalTok{BRAIN\_VOLUME }\SpecialCharTok{\textasciitilde{}}\NormalTok{ dfR}\SpecialCharTok{$}\NormalTok{AGE }\SpecialCharTok{+}\NormalTok{ dfR}\SpecialCharTok{$}\NormalTok{SEX }\SpecialCharTok{+} \FunctionTok{I}\NormalTok{(dfR}\SpecialCharTok{$}\NormalTok{AGE}\SpecialCharTok{\^{}}\DecValTok{2}\NormalTok{) }\SpecialCharTok{+} \FunctionTok{I}\NormalTok{(dfR}\SpecialCharTok{$}\NormalTok{AGE}\SpecialCharTok{\^{}}\DecValTok{3}\NormalTok{) }\SpecialCharTok{+} \FunctionTok{I}\NormalTok{(dfR}\SpecialCharTok{$}\NormalTok{AGE}\SpecialCharTok{\^{}}\DecValTok{4}\NormalTok{) }\SpecialCharTok{+} \FunctionTok{I}\NormalTok{(dfR}\SpecialCharTok{$}\NormalTok{AGE}\SpecialCharTok{\^{}}\DecValTok{5}\NormalTok{))}
\FunctionTok{anova}\NormalTok{(f1,f2,f3,f4,f5)}
\end{Highlighting}
\end{Shaded}

\begin{verbatim}
## Analysis of Variance Table
## 
## Model 1: dfR$BRAIN_VOLUME ~ dfR$AGE + dfR$SEX
## Model 2: dfR$BRAIN_VOLUME ~ dfR$AGE + dfR$SEX + I(dfR$AGE^2)
## Model 3: dfR$BRAIN_VOLUME ~ dfR$AGE + dfR$SEX + I(dfR$AGE^2) + I(dfR$AGE^3)
## Model 4: dfR$BRAIN_VOLUME ~ dfR$AGE + dfR$SEX + I(dfR$AGE^2) + I(dfR$AGE^3) + 
##     I(dfR$AGE^4)
## Model 5: dfR$BRAIN_VOLUME ~ dfR$AGE + dfR$SEX + I(dfR$AGE^2) + I(dfR$AGE^3) + 
##     I(dfR$AGE^4) + I(dfR$AGE^5)
##   Res.Df     RSS Df Sum of Sq      F  Pr(>F)  
## 1    259 2906017                              
## 2    258 2861992  1     44026 4.0217 0.04597 *
## 3    257 2846303  1     15689 1.4332 0.23236  
## 4    256 2794262  1     52040 4.7538 0.03015 *
## 5    255 2791479  1      2783 0.2543 0.61452  
## ---
## Signif. codes:  0 '***' 0.001 '**' 0.01 '*' 0.05 '.' 0.1 ' ' 1
\end{verbatim}

\#degree 2 is the better

\begin{Shaded}
\begin{Highlighting}[]
\FunctionTok{summary}\NormalTok{(f2) }
\end{Highlighting}
\end{Shaded}

\begin{verbatim}
## 
## Call:
## lm(formula = dfR$BRAIN_VOLUME ~ dfR$AGE + dfR$SEX + I(dfR$AGE^2))
## 
## Residuals:
##     Min      1Q  Median      3Q     Max 
## -255.59  -67.48    0.14   51.98  427.06 
## 
## Coefficients:
##               Estimate Std. Error t value Pr(>|t|)    
## (Intercept)  422.95693  418.29806   1.011   0.3129    
## dfR$AGE       23.70259   11.31683   2.094   0.0372 *  
## dfR$SEXMALE  200.74167   13.32028  15.070   <2e-16 ***
## I(dfR$AGE^2)  -0.15162    0.07611  -1.992   0.0474 *  
## ---
## Signif. codes:  0 '***' 0.001 '**' 0.01 '*' 0.05 '.' 0.1 ' ' 1
## 
## Residual standard error: 105.3 on 258 degrees of freedom
## Multiple R-squared:  0.4861, Adjusted R-squared:  0.4801 
## F-statistic: 81.34 on 3 and 258 DF,  p-value: < 2.2e-16
\end{verbatim}

Cada una de las pendientes de un modelo de regresión lineal múltiple se
define del siguiente modo: Si el resto de variables se mantienen
constantes, por cada unidad que aumenta el predictor en cuestión, la
variable Y varía en promedio tantas unidades como indica la pendiente.
En el caso del predictor AGE, si el resto de variables no varían, por
cada unidad de AGE que aumenta el volumen cerebral se aumenta en
promedio 23.70259 unidades. Las variables son significativas (al menos
`*' 0.05), El coeficiente de determinacion R\^{}2(0.4982 o 49.82\%): es
el porcentaje de la variación en la variable de respuesta que es
explicado por un modelo lineal. El R-cuadrado siempre está entre 0 y
100\%:

0\% indica que el modelo no explica ninguna porción de la variabilidad
de los datos de respuesta en torno a su media. 100\% indica que el
modelo explica toda la variabilidad de los datos de respuesta en torno a
su media.

\begin{Shaded}
\begin{Highlighting}[]
\FunctionTok{plot.new}\NormalTok{()}
\FunctionTok{plot}\NormalTok{(}\FunctionTok{fitted}\NormalTok{(f2), }\FunctionTok{residuals}\NormalTok{(f2), }\AttributeTok{xlab =} \StringTok{"Approached values"}\NormalTok{,  }\AttributeTok{ylab =} \StringTok{"Residuals"}\NormalTok{)}
  \FunctionTok{abline}\NormalTok{(}\AttributeTok{h=}\DecValTok{0}\NormalTok{, }\AttributeTok{lty=}\DecValTok{2}\NormalTok{)}
    \FunctionTok{lines}\NormalTok{(}\FunctionTok{smooth.spline}\NormalTok{(}\FunctionTok{fitted}\NormalTok{(f2), }\FunctionTok{residuals}\NormalTok{(f2)))}
\end{Highlighting}
\end{Shaded}

\includegraphics{ProyectoContrastesRegresion_files/figure-latex/unnamed-chunk-25-1.pdf}

\begin{Shaded}
\begin{Highlighting}[]
\FunctionTok{plot}\NormalTok{(f2}\SpecialCharTok{$}\NormalTok{residuals)}
\end{Highlighting}
\end{Shaded}

\includegraphics{ProyectoContrastesRegresion_files/figure-latex/unnamed-chunk-26-1.pdf}

\begin{Shaded}
\begin{Highlighting}[]
\FunctionTok{summary}\NormalTok{(f2}\SpecialCharTok{$}\NormalTok{residuals)}
\end{Highlighting}
\end{Shaded}

\begin{verbatim}
##      Min.   1st Qu.    Median      Mean   3rd Qu.      Max. 
## -255.5923  -67.4764    0.1357    0.0000   51.9849  427.0647
\end{verbatim}

\begin{Shaded}
\begin{Highlighting}[]
\FunctionTok{boxplot}\NormalTok{(f2}\SpecialCharTok{$}\NormalTok{residuals)}
\end{Highlighting}
\end{Shaded}

\includegraphics{ProyectoContrastesRegresion_files/figure-latex/unnamed-chunk-26-2.pdf}

\begin{center}\rule{0.5\linewidth}{0.5pt}\end{center}

\#Caret

Caret actúa como interfaz única para otros muchos métodos. En total,
hace de interfaz con 136 métodos de regresión distintos.

\begin{Shaded}
\begin{Highlighting}[]
\CommentTok{\#install.packages("caret")}
\FunctionTok{library}\NormalTok{(caret)}
\end{Highlighting}
\end{Shaded}

\begin{verbatim}
## Loading required package: lattice
\end{verbatim}

\begin{verbatim}
## 
## Attaching package: 'caret'
\end{verbatim}

\begin{verbatim}
## The following object is masked from 'package:purrr':
## 
##     lift
\end{verbatim}

\begin{Shaded}
\begin{Highlighting}[]
\FunctionTok{modelLookup}\NormalTok{()}
\end{Highlighting}
\end{Shaded}

\begin{verbatim}
##                   model           parameter
## 1                   ada                iter
## 2                   ada            maxdepth
## 3                   ada                  nu
## 4                AdaBag              mfinal
## 5                AdaBag            maxdepth
## 9              adaboost               nIter
## 10             adaboost              method
## 6           AdaBoost.M1              mfinal
## 7           AdaBoost.M1            maxdepth
## 8           AdaBoost.M1           coeflearn
## 11                amdai               model
## 12                ANFIS          num.labels
## 13                ANFIS            max.iter
## 14               avNNet                size
## 15               avNNet               decay
## 16               avNNet                 bag
## 17                 awnb              smooth
## 18                awtan               score
## 19                awtan              smooth
## 20                  bag                vars
## 21             bagEarth              nprune
## 22             bagEarth              degree
## 23          bagEarthGCV              degree
## 24               bagFDA              degree
## 25               bagFDA              nprune
## 26            bagFDAGCV              degree
## 27                  bam              select
## 28                  bam              method
## 29          bartMachine           num_trees
## 30          bartMachine                   k
## 31          bartMachine               alpha
## 32          bartMachine                beta
## 33          bartMachine                  nu
## 34             bayesglm           parameter
## 35                binda        lambda.freqs
## 36           blackboost               mstop
## 37           blackboost            maxdepth
## 38               blasso            sparsity
## 39       blassoAveraged           parameter
## 40               bridge           parameter
## 41                 brnn             neurons
## 42                BstLm               mstop
## 43                BstLm                  nu
## 44                bstSm               mstop
## 45                bstSm                  nu
## 46              bstTree               mstop
## 47              bstTree            maxdepth
## 48              bstTree                  nu
## 49                 C5.0              trials
## 50                 C5.0               model
## 51                 C5.0              winnow
## 52             C5.0Cost              trials
## 53             C5.0Cost               model
## 54             C5.0Cost              winnow
## 55             C5.0Cost                cost
## 56            C5.0Rules           parameter
## 57             C5.0Tree           parameter
## 58              cforest                mtry
## 59                chaid              alpha2
## 60                chaid              alpha3
## 61                chaid              alpha4
## 62               CSimca           parameter
## 63                ctree        mincriterion
## 64               ctree2            maxdepth
## 65               ctree2        mincriterion
## 66               cubist          committees
## 67               cubist           neighbors
## 68                  dda               model
## 69                  dda           shrinkage
## 70            deepboost            num_iter
## 71            deepboost          tree_depth
## 72            deepboost                beta
## 73            deepboost              lambda
## 74            deepboost           loss_type
## 75               DENFIS                Dthr
## 76               DENFIS            max.iter
## 77                  dnn              layer1
## 78                  dnn              layer2
## 79                  dnn              layer3
## 80                  dnn      hidden_dropout
## 81                  dnn     visible_dropout
## 82            dwdLinear              lambda
## 83            dwdLinear                qval
## 84              dwdPoly              lambda
## 85              dwdPoly                qval
## 86              dwdPoly              degree
## 87              dwdPoly               scale
## 88            dwdRadial              lambda
## 89            dwdRadial                qval
## 90            dwdRadial               sigma
## 91                earth              nprune
## 92                earth              degree
## 93                  elm                nhid
## 94                  elm              actfun
## 95                 enet            fraction
## 96                 enet              lambda
## 97               evtree               alpha
## 98           extraTrees                mtry
## 99           extraTrees       numRandomCuts
## 100                 fda              degree
## 101                 fda              nprune
## 102             FH.GBML        max.num.rule
## 103             FH.GBML           popu.size
## 104             FH.GBML             max.gen
## 105              FIR.DM          num.labels
## 106              FIR.DM            max.iter
## 107                foba                   k
## 108                foba              lambda
## 109           FRBCS.CHI          num.labels
## 110           FRBCS.CHI             type.mf
## 111             FRBCS.W          num.labels
## 112             FRBCS.W             type.mf
## 113              FS.HGD          num.labels
## 114              FS.HGD            max.iter
## 115                 gam              select
## 116                 gam              method
## 117            gamboost               mstop
## 118            gamboost               prune
## 119            gamLoess                span
## 120            gamLoess              degree
## 121           gamSpline                  df
## 122       gaussprLinear           parameter
## 123         gaussprPoly              degree
## 124         gaussprPoly               scale
## 125       gaussprRadial               sigma
## 131                 gbm             n.trees
## 132                 gbm   interaction.depth
## 133                 gbm           shrinkage
## 134                 gbm      n.minobsinnode
## 126             gbm_h2o              ntrees
## 127             gbm_h2o           max_depth
## 128             gbm_h2o            min_rows
## 129             gbm_h2o          learn_rate
## 130             gbm_h2o     col_sample_rate
## 135            gcvEarth              degree
## 136        GFS.FR.MOGUL             max.gen
## 137        GFS.FR.MOGUL            max.iter
## 138        GFS.FR.MOGUL            max.tune
## 139           GFS.LT.RS           popu.size
## 140           GFS.LT.RS          num.labels
## 141           GFS.LT.RS             max.gen
## 142          GFS.THRIFT           popu.size
## 143          GFS.THRIFT          num.labels
## 144          GFS.THRIFT             max.gen
## 146                 glm           parameter
## 145              glm.nb                link
## 147            glmboost               mstop
## 148            glmboost               prune
## 151              glmnet               alpha
## 152              glmnet              lambda
## 149          glmnet_h2o               alpha
## 150          glmnet_h2o              lambda
## 153          glmStepAIC           parameter
## 154                gpls              K.prov
## 155                 hda               gamma
## 156                 hda              lambda
## 157                 hda              newdim
## 158                hdda           threshold
## 159                hdda               model
## 160               hdrda               gamma
## 161               hdrda              lambda
## 162               hdrda      shrinkage_type
## 163               HYFIS          num.labels
## 164               HYFIS            max.iter
## 165                 icr              n.comp
## 166                 J48                   C
## 167                 J48                   M
## 168                JRip              NumOpt
## 169                JRip            NumFolds
## 170                JRip          MinWeights
## 171           kernelpls               ncomp
## 172                kknn                kmax
## 173                kknn            distance
## 174                kknn              kernel
## 175                 knn                   k
## 176            krlsPoly              lambda
## 177            krlsPoly              degree
## 178          krlsRadial              lambda
## 179          krlsRadial               sigma
## 180                lars            fraction
## 181               lars2                step
## 182               lasso            fraction
## 183                 lda           parameter
## 184                lda2               dimen
## 185        leapBackward               nvmax
## 186         leapForward               nvmax
## 187             leapSeq               nvmax
## 188               Linda           parameter
## 189                  lm           intercept
## 190           lmStepAIC           parameter
## 191                 LMT                iter
## 192              loclda                   k
## 193            logicBag             nleaves
## 194            logicBag              ntrees
## 195          LogitBoost               nIter
## 196              logreg            treesize
## 197              logreg              ntrees
## 198         lssvmLinear                 tau
## 199           lssvmPoly              degree
## 200           lssvmPoly               scale
## 201           lssvmPoly                 tau
## 202         lssvmRadial               sigma
## 203         lssvmRadial                 tau
## 204                 lvq                size
## 205                 lvq                   k
## 206                  M5              pruned
## 207                  M5            smoothed
## 208                  M5               rules
## 209             M5Rules              pruned
## 210             M5Rules            smoothed
## 211                manb              smooth
## 212                manb               prior
## 213                 mda          subclasses
## 214                Mlda           parameter
## 215                 mlp                size
## 216       mlpKerasDecay                size
## 217       mlpKerasDecay              lambda
## 218       mlpKerasDecay          batch_size
## 219       mlpKerasDecay                  lr
## 220       mlpKerasDecay                 rho
## 221       mlpKerasDecay               decay
## 222       mlpKerasDecay          activation
## 223   mlpKerasDecayCost                size
## 224   mlpKerasDecayCost              lambda
## 225   mlpKerasDecayCost          batch_size
## 226   mlpKerasDecayCost                  lr
## 227   mlpKerasDecayCost                 rho
## 228   mlpKerasDecayCost               decay
## 229   mlpKerasDecayCost                cost
## 230   mlpKerasDecayCost          activation
## 231     mlpKerasDropout                size
## 232     mlpKerasDropout             dropout
## 233     mlpKerasDropout          batch_size
## 234     mlpKerasDropout                  lr
## 235     mlpKerasDropout                 rho
## 236     mlpKerasDropout               decay
## 237     mlpKerasDropout          activation
## 238 mlpKerasDropoutCost                size
## 239 mlpKerasDropoutCost             dropout
## 240 mlpKerasDropoutCost          batch_size
## 241 mlpKerasDropoutCost                  lr
## 242 mlpKerasDropoutCost                 rho
## 243 mlpKerasDropoutCost               decay
## 244 mlpKerasDropoutCost                cost
## 245 mlpKerasDropoutCost          activation
## 246               mlpML              layer1
## 247               mlpML              layer2
## 248               mlpML              layer3
## 249              mlpSGD                size
## 250              mlpSGD               l2reg
## 251              mlpSGD              lambda
## 252              mlpSGD          learn_rate
## 253              mlpSGD            momentum
## 254              mlpSGD               gamma
## 255              mlpSGD         minibatchsz
## 256              mlpSGD             repeats
## 257      mlpWeightDecay                size
## 258      mlpWeightDecay               decay
## 259    mlpWeightDecayML              layer1
## 260    mlpWeightDecayML              layer2
## 261    mlpWeightDecayML              layer3
## 262    mlpWeightDecayML               decay
## 263              monmlp             hidden1
## 264              monmlp          n.ensemble
## 265             msaenet              alphas
## 266             msaenet              nsteps
## 267             msaenet               scale
## 268            multinom               decay
## 269               mxnet              layer1
## 270               mxnet              layer2
## 271               mxnet              layer3
## 272               mxnet       learning.rate
## 273               mxnet            momentum
## 274               mxnet             dropout
## 275               mxnet          activation
## 276           mxnetAdam              layer1
## 277           mxnetAdam              layer2
## 278           mxnetAdam              layer3
## 279           mxnetAdam             dropout
## 280           mxnetAdam               beta1
## 281           mxnetAdam               beta2
## 282           mxnetAdam        learningrate
## 283           mxnetAdam          activation
## 284         naive_bayes             laplace
## 285         naive_bayes           usekernel
## 286         naive_bayes              adjust
## 287                  nb                  fL
## 288                  nb           usekernel
## 289                  nb              adjust
## 290          nbDiscrete              smooth
## 291            nbSearch                   k
## 292            nbSearch             epsilon
## 293            nbSearch              smooth
## 294            nbSearch        final_smooth
## 295            nbSearch           direction
## 296           neuralnet              layer1
## 297           neuralnet              layer2
## 298           neuralnet              layer3
## 299                nnet                size
## 300                nnet               decay
## 301                nnls           parameter
## 302         nodeHarvest            maxinter
## 303         nodeHarvest                mode
## 304                null           parameter
## 305                OneR           parameter
## 306          ordinalNet               alpha
## 307          ordinalNet            criteria
## 308          ordinalNet                link
## 309           ordinalRF               nsets
## 310           ordinalRF         ntreeperdiv
## 311           ordinalRF          ntreefinal
## 312              ORFlog                mtry
## 313              ORFpls                mtry
## 314            ORFridge                mtry
## 315              ORFsvm                mtry
## 316                ownn                   K
## 317                 pam           threshold
## 318               parRF                mtry
## 319                PART           threshold
## 320                PART              pruned
## 321             partDSA      cut.off.growth
## 322             partDSA                 MPD
## 323             pcaNNet                size
## 324             pcaNNet               decay
## 325                 pcr               ncomp
## 326                 pda              lambda
## 327                pda2                  df
## 328           penalized             lambda1
## 329           penalized             lambda2
## 330        PenalizedLDA              lambda
## 331        PenalizedLDA                   K
## 332                 plr              lambda
## 333                 plr                  cp
## 334                 pls               ncomp
## 335             plsRglm                  nt
## 336             plsRglm   alpha.pvals.expli
## 337                polr              method
## 338                 ppr              nterms
## 339                PRIM          peel.alpha
## 340                PRIM         paste.alpha
## 341                PRIM            mass.min
## 342          protoclass                 eps
## 343          protoclass           Minkowski
## 344                 qda           parameter
## 345              QdaCov           parameter
## 346                 qrf                mtry
## 347                qrnn            n.hidden
## 348                qrnn             penalty
## 349                qrnn                 bag
## 350           randomGLM maxInteractionOrder
## 351              ranger                mtry
## 352              ranger           splitrule
## 353              ranger       min.node.size
## 354                 rbf                size
## 355              rbfDDA   negativeThreshold
## 356             Rborist           predFixed
## 357             Rborist             minNode
## 358                 rda               gamma
## 359                 rda              lambda
## 360         regLogistic                cost
## 361         regLogistic                loss
## 362         regLogistic             epsilon
## 363              relaxo              lambda
## 364              relaxo                 phi
## 365                  rf                mtry
## 366              rFerns               depth
## 367               RFlda                   q
## 368             rfRules                mtry
## 369             rfRules            maxdepth
## 370               ridge              lambda
## 371                rlda           estimator
## 372                 rlm           intercept
## 373                 rlm                 psi
## 374                rmda                   K
## 375                rmda               model
## 376                rocc              xgenes
## 377      rotationForest                   K
## 378      rotationForest                   L
## 379    rotationForestCp                   K
## 380    rotationForestCp                   L
## 381    rotationForestCp                  cp
## 382               rpart                  cp
## 383            rpart1SE           parameter
## 384              rpart2            maxdepth
## 385           rpartCost                  cp
## 386           rpartCost                Cost
## 387          rpartScore                  cp
## 388          rpartScore               split
## 389          rpartScore               prune
## 390             rqlasso              lambda
## 391                rqnc              lambda
## 392                rqnc             penalty
## 393                 RRF                mtry
## 394                 RRF             coefReg
## 395                 RRF             coefImp
## 396           RRFglobal                mtry
## 397           RRFglobal             coefReg
## 398               rrlda              lambda
## 399               rrlda                  hp
## 400               rrlda             penalty
## 401              RSimca           parameter
## 402           rvmLinear           parameter
## 403             rvmPoly               scale
## 404             rvmPoly              degree
## 405           rvmRadial               sigma
## 406                 SBC                 r.a
## 407                 SBC            eps.high
## 408                 SBC             eps.low
## 409                 sda            diagonal
## 410                 sda              lambda
## 411                sdwd              lambda
## 412                sdwd             lambda2
## 413              simpls               ncomp
## 414               SLAVE          num.labels
## 415               SLAVE            max.iter
## 416               SLAVE             max.gen
## 417                slda           parameter
## 418                smda             NumVars
## 419                smda              lambda
## 420                smda                   R
## 421                 snn              lambda
## 422           sparseLDA             NumVars
## 423           sparseLDA              lambda
## 424           spikeslab                vars
## 425                spls                   K
## 426                spls                 eta
## 427                spls               kappa
## 428             stepLDA              maxvar
## 429             stepLDA           direction
## 430             stepQDA              maxvar
## 431             stepQDA           direction
## 432             superpc           threshold
## 433             superpc        n.components
## 434 svmBoundrangeString              length
## 435 svmBoundrangeString                   C
## 436       svmExpoString              lambda
## 437       svmExpoString                   C
## 438           svmLinear                   C
## 439          svmLinear2                cost
## 440          svmLinear3                cost
## 441          svmLinear3                Loss
## 442    svmLinearWeights                cost
## 443    svmLinearWeights              weight
## 444   svmLinearWeights2                cost
## 445   svmLinearWeights2                Loss
## 446   svmLinearWeights2              weight
## 447             svmPoly              degree
## 448             svmPoly               scale
## 449             svmPoly                   C
## 450           svmRadial               sigma
## 451           svmRadial                   C
## 452       svmRadialCost                   C
## 453      svmRadialSigma               sigma
## 454      svmRadialSigma                   C
## 455    svmRadialWeights               sigma
## 456    svmRadialWeights                   C
## 457    svmRadialWeights              Weight
## 458   svmSpectrumString              length
## 459   svmSpectrumString                   C
## 460                 tan               score
## 461                 tan              smooth
## 462           tanSearch                   k
## 463           tanSearch             epsilon
## 464           tanSearch              smooth
## 465           tanSearch        final_smooth
## 466           tanSearch                  sp
## 467             treebag           parameter
## 468          vbmpRadial       estimateTheta
## 469          vglmAdjCat            parallel
## 470          vglmAdjCat                link
## 471       vglmContRatio            parallel
## 472       vglmContRatio                link
## 473      vglmCumulative            parallel
## 474      vglmCumulative                link
## 475       widekernelpls               ncomp
## 476                  WM          num.labels
## 477                  WM             type.mf
## 478                wsrf                mtry
## 479             xgbDART             nrounds
## 480             xgbDART           max_depth
## 481             xgbDART                 eta
## 482             xgbDART               gamma
## 483             xgbDART           subsample
## 484             xgbDART    colsample_bytree
## 485             xgbDART           rate_drop
## 486             xgbDART           skip_drop
## 487             xgbDART    min_child_weight
## 488           xgbLinear             nrounds
## 489           xgbLinear              lambda
## 490           xgbLinear               alpha
## 491           xgbLinear                 eta
## 492             xgbTree             nrounds
## 493             xgbTree           max_depth
## 494             xgbTree                 eta
## 495             xgbTree               gamma
## 496             xgbTree    colsample_bytree
## 497             xgbTree    min_child_weight
## 498             xgbTree           subsample
## 499                 xyf                xdim
## 500                 xyf                ydim
## 501                 xyf        user.weights
## 502                 xyf                topo
##                                                                                       label
## 1                                                                                    #Trees
## 2                                                                            Max Tree Depth
## 3                                                                             Learning Rate
## 4                                                                                    #Trees
## 5                                                                            Max Tree Depth
## 9                                                                                    #Trees
## 10                                                                                   Method
## 6                                                                                    #Trees
## 7                                                                            Max Tree Depth
## 8                                                                          Coefficient Type
## 11                                                                               Model Type
## 12                                                                             #Fuzzy Terms
## 13                                                                          Max. Iterations
## 14                                                                            #Hidden Units
## 15                                                                             Weight Decay
## 16                                                                                  Bagging
## 17                                                                      Smoothing Parameter
## 18                                                                           Score Function
## 19                                                                      Smoothing Parameter
## 20                                                            #Randomly Selected Predictors
## 21                                                                                   #Terms
## 22                                                                           Product Degree
## 23                                                                           Product Degree
## 24                                                                           Product Degree
## 25                                                                                   #Terms
## 26                                                                           Product Degree
## 27                                                                        Feature Selection
## 28                                                                                   Method
## 29                                                                                   #Trees
## 30                                                                           Prior Boundary
## 31                                                        Base Terminal Node Hyperparameter
## 32                                                       Power Terminal Node Hyperparameter
## 33                                                                       Degrees of Freedom
## 34                                                                                parameter
## 35                                                                      Shrinkage Intensity
## 36                                                                                   #Trees
## 37                                                                           Max Tree Depth
## 38                                                                       Sparsity Threshold
## 39                                                                                parameter
## 40                                                                                parameter
## 41                                                                                # Neurons
## 42                                                                    # Boosting Iterations
## 43                                                                                Shrinkage
## 44                                                                    # Boosting Iterations
## 45                                                                                Shrinkage
## 46                                                                    # Boosting Iterations
## 47                                                                           Max Tree Depth
## 48                                                                                Shrinkage
## 49                                                                    # Boosting Iterations
## 50                                                                               Model Type
## 51                                                                                   Winnow
## 52                                                                    # Boosting Iterations
## 53                                                                               Model Type
## 54                                                                                   Winnow
## 55                                                                                     Cost
## 56                                                                                     none
## 57                                                                                     none
## 58                                                            #Randomly Selected Predictors
## 59                                                                        Merging Threshold
## 60                                                        Splitting former Merged Threshold
## 61  \n                                                    Splitting former Merged Threshold
## 62                                                                                parameter
## 63                                                                    1 - P-Value Threshold
## 64                                                                           Max Tree Depth
## 65                                                                    1 - P-Value Threshold
## 66                                                                              #Committees
## 67                                                                               #Instances
## 68                                                                                    Model
## 69                                                                           Shrinkage Type
## 70                                                                    # Boosting Iterations
## 71                                                                               Tree Depth
## 72                                                                        L1 Regularization
## 73                                                                Tree Depth Regularization
## 74                                                                                     Loss
## 75                                                                                Threshold
## 76                                                                          Max. Iterations
## 77                                                                           Hidden Layer 1
## 78                                                                           Hidden Layer 2
## 79                                                                           Hidden Layer 3
## 80                                                                          Hidden Dropouts
## 81                                                                          Visible Dropout
## 82                                                                 Regularization Parameter
## 83                                                                                        q
## 84                                                                 Regularization Parameter
## 85                                                                                        q
## 86                                                                        Polynomial Degree
## 87                                                                                    Scale
## 88                                                                 Regularization Parameter
## 89                                                                                        q
## 90                                                                                    Sigma
## 91                                                                                   #Terms
## 92                                                                           Product Degree
## 93                                                                            #Hidden Units
## 94                                                                      Activation Function
## 95                                                                Fraction of Full Solution
## 96                                                                             Weight Decay
## 97                                                                     Complexity Parameter
## 98                                                           # Randomly Selected Predictors
## 99                                                                            # Random Cuts
## 100                                                                          Product Degree
## 101                                                                                  #Terms
## 102                                                                             Max. #Rules
## 103                                                                         Population Size
## 104                                                                        Max. Generations
## 105                                                                            #Fuzzy Terms
## 106                                                                         Max. Iterations
## 107                                                                     #Variables Retained
## 108                                                                              L2 Penalty
## 109                                                                            #Fuzzy Terms
## 110                                                                     Membership Function
## 111                                                                            #Fuzzy Terms
## 112                                                                     Membership Function
## 113                                                                            #Fuzzy Terms
## 114                                                                         Max. Iterations
## 115                                                                       Feature Selection
## 116                                                                                  Method
## 117                                                                   # Boosting Iterations
## 118                                                                              AIC Prune?
## 119                                                                                    Span
## 120                                                                                  Degree
## 121                                                                      Degrees of Freedom
## 122                                                                               Parameter
## 123                                                                       Polynomial Degree
## 124                                                                                   Scale
## 125                                                                                   Sigma
## 131                                                                   # Boosting Iterations
## 132                                                                          Max Tree Depth
## 133                                                                               Shrinkage
## 134                                                                 Min. Terminal Node Size
## 126                                                                   # Boosting Iterations
## 127                                                                          Max Tree Depth
## 128                                                                 Min. Terminal Node Size
## 129                                                                               Shrinkage
## 130                                                           #Randomly Selected Predictors
## 135                                                                          Product Degree
## 136                                                                        Max. Generations
## 137                                                                         Max. Iterations
## 138                                                                  Max. Tuning Iterations
## 139                                                                         Population Size
## 140                                                                          # Fuzzy Labels
## 141                                                                        Max. Generations
## 142                                                                         Population Size
## 143                                                                          # Fuzzy Labels
## 144                                                                        Max. Generations
## 146                                                                               parameter
## 145                                                                           Link Function
## 147                                                                   # Boosting Iterations
## 148                                                                              AIC Prune?
## 151                                                                       Mixing Percentage
## 152                                                                Regularization Parameter
## 149                                                                       Mixing Percentage
## 150                                                                Regularization Parameter
## 153                                                                               parameter
## 154                                                                             #Components
## 155                                                                                   Gamma
## 156                                                                                  Lambda
## 157                                                Dimension of the Discriminative Subspace
## 158                                                                               Threshold
## 159                                                                              Model Type
## 160                                                                                   Gamma
## 161                                                                                  Lambda
## 162                                                                          Shrinkage Type
## 163                                                                            #Fuzzy Terms
## 164                                                                         Max. Iterations
## 165                                                                             #Components
## 166                                                                    Confidence Threshold
## 167                                                              Minimum Instances Per Leaf
## 168                                                                         # Optimizations
## 169                                                                                 # Folds
## 170                                                                             Min Weights
## 171                                                                             #Components
## 172                                                                         Max. #Neighbors
## 173                                                                                Distance
## 174                                                                                  Kernel
## 175                                                                              #Neighbors
## 176                                                                Regularization Parameter
## 177                                                                       Polynomial Degree
## 178                                                                Regularization Parameter
## 179                                                                                   Sigma
## 180                                                                                Fraction
## 181                                                                                  #Steps
## 182                                                               Fraction of Full Solution
## 183                                                                               parameter
## 184                                                                 #Discriminant Functions
## 185                                                            Maximum Number of Predictors
## 186                                                            Maximum Number of Predictors
## 187                                                            Maximum Number of Predictors
## 188                                                                                    none
## 189                                                                               intercept
## 190                                                                               parameter
## 191                                                                             # Iteratons
## 192                                                                      #Nearest Neighbors
## 193                                                                Maximum Number of Leaves
## 194                                                                         Number of Trees
## 195                                                                   # Boosting Iterations
## 196                                                                Maximum Number of Leaves
## 197                                                                         Number of Trees
## 198                                                                Regularization Parameter
## 199                                                                       Polynomial Degree
## 200                                                                                   Scale
## 201                                                                Regularization Parameter
## 202                                                                                   Sigma
## 203                                                                Regularization Parameter
## 204                                                                           Codebook Size
## 205                                                                             #Prototypes
## 206                                                                                  Pruned
## 207                                                                                Smoothed
## 208                                                                                   Rules
## 209                                                                                  Pruned
## 210                                                                                Smoothed
## 211                                                                     Smoothing Parameter
## 212                                                                       Prior Probability
## 213                                                                   #Subclasses Per Class
## 214                                                                               parameter
## 215                                                                           #Hidden Units
## 216                                                                           #Hidden Units
## 217                                                                       L2 Regularization
## 218                                                                              Batch Size
## 219                                                                           Learning Rate
## 220                                                                                     Rho
## 221                                                                     Learning Rate Decay
## 222                                                                     Activation Function
## 223                                                                           #Hidden Units
## 224                                                                       L2 Regularization
## 225                                                                              Batch Size
## 226                                                                           Learning Rate
## 227                                                                                     Rho
## 228                                                                     Learning Rate Decay
## 229                                                                                    Cost
## 230                                                                     Activation Function
## 231                                                                           #Hidden Units
## 232                                                                            Dropout Rate
## 233                                                                              Batch Size
## 234                                                                           Learning Rate
## 235                                                                                     Rho
## 236                                                                     Learning Rate Decay
## 237                                                                     Activation Function
## 238                                                                           #Hidden Units
## 239                                                                            Dropout Rate
## 240                                                                              Batch Size
## 241                                                                           Learning Rate
## 242                                                                                     Rho
## 243                                                                     Learning Rate Decay
## 244                                                                                    Cost
## 245                                                                     Activation Function
## 246                                                                    #Hidden Units layer1
## 247                                                                    #Hidden Units layer2
## 248                                                                    #Hidden Units layer3
## 249                                                                           #Hidden Units
## 250                                                                       L2 Regularization
## 251                                                                   RMSE Gradient Scaling
## 252                                                                           Learning Rate
## 253                                                                                Momentum
## 254                                                                     Learning Rate Decay
## 255                                                                              Batch Size
## 256                                                                                 #Models
## 257                                                                           #Hidden Units
## 258                                                                            Weight Decay
## 259                                                                    #Hidden Units layer1
## 260                                                                    #Hidden Units layer2
## 261                                                                    #Hidden Units layer3
## 262                                                                            Weight Decay
## 263                                                                           #Hidden Units
## 264                                                                                 #Models
## 265                                                                                   Alpha
## 266                                                              #Adaptive Estimation Steps
## 267                                                          Adaptive Weight Scaling Factor
## 268                                                                            Weight Decay
## 269                                                                #Hidden Units in Layer 1
## 270                                                                #Hidden Units in Layer 2
## 271                                                                #Hidden Units in Layer 3
## 272                                                                           Learning Rate
## 273                                                                                Momentum
## 274                                                                            Dropout Rate
## 275                                                                     Activation Function
## 276                                                                #Hidden Units in Layer 1
## 277                                                                #Hidden Units in Layer 2
## 278                                                                #Hidden Units in Layer 3
## 279                                                                            Dropout Rate
## 280                                                                                   beta1
## 281                                                                                   beta2
## 282                                                                           Learning Rate
## 283                                                                     Activation Function
## 284                                                                      Laplace Correction
## 285                                                                       Distribution Type
## 286                                                                    Bandwidth Adjustment
## 287                                                                      Laplace Correction
## 288                                                                       Distribution Type
## 289                                                                    Bandwidth Adjustment
## 290                                                                     Smoothing Parameter
## 291                                                                                  #Folds
## 292                                                            Minimum Absolute Improvement
## 293                                                                     Smoothing Parameter
## 294                                                               Final Smoothing Parameter
## 295                                                                        Search Direction
## 296                                                                #Hidden Units in Layer 1
## 297                                                                #Hidden Units in Layer 2
## 298                                                                #Hidden Units in Layer 3
## 299                                                                           #Hidden Units
## 300                                                                            Weight Decay
## 301                                                                               parameter
## 302                                                               Maximum Interaction Depth
## 303                                                                         Prediction Mode
## 304                                                                               parameter
## 305                                                                                    none
## 306                                                                       Mixing Percentage
## 307                                                                     Selection Criterion
## 308                                                                           Link Function
## 309                                           # score sets tried prior to the approximation
## 310                                                                  # of trees (small RFs)
## 311                                                                   # of trees (final RF)
## 312                                                           #Randomly Selected Predictors
## 313                                                           #Randomly Selected Predictors
## 314                                                           #Randomly Selected Predictors
## 315                                                           #Randomly Selected Predictors
## 316                                                                              #Neighbors
## 317                                                                     Shrinkage Threshold
## 318                                                           #Randomly Selected Predictors
## 319                                                                    Confidence Threshold
## 320                                                                                 Pruning
## 321                                                           Number of Terminal Partitions
## 322                                                              Minimum Percent Difference
## 323                                                                           #Hidden Units
## 324                                                                            Weight Decay
## 325                                                                             #Components
## 326                                                           Shrinkage Penalty Coefficient
## 327                                                                      Degrees of Freedom
## 328                                                                              L1 Penalty
## 329                                                                              L2 Penalty
## 330                                                                              L1 Penalty
## 331                                                                 #Discriminant Functions
## 332                                                                              L2 Penalty
## 333                                                                    Complexity Parameter
## 334                                                                             #Components
## 335                                                                         #PLS Components
## 336                                                                       p-Value threshold
## 337                                                                               parameter
## 338                                                                                 # Terms
## 339                                                                        peeling quantile
## 340                                                                        pasting quantile
## 341                                                                            minimum mass
## 342                                                                               Ball Size
## 343                                                                          Distance Order
## 344                                                                               parameter
## 345                                                                               parameter
## 346                                                           #Randomly Selected Predictors
## 347                                                                           #Hidden Units
## 348                                                                            Weight Decay
## 349                                                                          Bagged Models?
## 350                                                                       Interaction Order
## 351                                                           #Randomly Selected Predictors
## 352                                                                          Splitting Rule
## 353                                                                       Minimal Node Size
## 354                                                                           #Hidden Units
## 355                                                Activation Limit for Conflicting Classes
## 356                                                           #Randomly Selected Predictors
## 357                                                                       Minimal Node Size
## 358                                                                                   Gamma
## 359                                                                                  Lambda
## 360                                                                                    Cost
## 361                                                                           Loss Function
## 362                                                                               Tolerance
## 363                                                                       Penalty Parameter
## 364                                                                    Relaxation Parameter
## 365                                                           #Randomly Selected Predictors
## 366                                                                              Fern Depth
## 367                                                                               # Factors
## 368                                                           #Randomly Selected Predictors
## 369                                                                      Maximum Rule Depth
## 370                                                                            Weight Decay
## 371                                                                   Regularization Method
## 372                                                                               intercept
## 373                                                                                     psi
## 374                                                                   #Subclasses Per Class
## 375                                                                                   Model
## 376                                                                     #Variables Retained
## 377                                                                       #Variable Subsets
## 378                                                                           Ensemble Size
## 379                                                                       #Variable Subsets
## 380                                                                           Ensemble Size
## 381                                                                    Complexity Parameter
## 382                                                                    Complexity Parameter
## 383                                                                               parameter
## 384                                                                          Max Tree Depth
## 385                                                                    Complexity Parameter
## 386                                                                                    Cost
## 387                                                                    Complexity Parameter
## 388                                                                          Split Function
## 389                                                                         Pruning Measure
## 390                                                                              L1 Penalty
## 391                                                                              L1 Penalty
## 392                                                                            Penalty Type
## 393                                                           #Randomly Selected Predictors
## 394                                                                    Regularization Value
## 395                                                                  Importance Coefficient
## 396                                                           #Randomly Selected Predictors
## 397                                                                    Regularization Value
## 398                                                                       Penalty Parameter
## 399                                                                    Robustness Parameter
## 400                                                                            Penalty Type
## 401                                                                               parameter
## 402                                                                               parameter
## 403                                                                                   Scale
## 404                                                                       Polynomial Degree
## 405                                                                                   Sigma
## 406                                                                                  Radius
## 407                                                                         Upper Threshold
## 408                                                                         Lower Threshold
## 409                                                                             Diagonalize
## 410                                                                               shrinkage
## 411                                                                              L1 Penalty
## 412                                                                              L2 Penalty
## 413                                                                             #Components
## 414                                                                            #Fuzzy Terms
## 415                                                                         Max. Iterations
## 416                                                                        Max. Generations
## 417                                                                                    none
## 418                                                                            # Predictors
## 419                                                                                  Lambda
## 420                                                                            # Subclasses
## 421                                                                 Stabilization Parameter
## 422                                                                            # Predictors
## 423                                                                                  Lambda
## 424                                                                      Variables Retained
## 425                                                                             #Components
## 426                                                                               Threshold
## 427                                                                                   Kappa
## 428                                                                      Maximum #Variables
## 429                                                                        Search Direction
## 430                                                                      Maximum #Variables
## 431                                                                        Search Direction
## 432                                                                               Threshold
## 433                                                                             #Components
## 434                                                                                  length
## 435                                                                                    Cost
## 436                                                                                  lambda
## 437                                                                                    Cost
## 438                                                                                    Cost
## 439                                                                                    Cost
## 440                                                                                    Cost
## 441                                                                           Loss Function
## 442                                                                                    Cost
## 443                                                                            Class Weight
## 444                                                                                    Cost
## 445                                                                           Loss Function
## 446                                                                            Class Weight
## 447                                                                       Polynomial Degree
## 448                                                                                   Scale
## 449                                                                                    Cost
## 450                                                                                   Sigma
## 451                                                                                    Cost
## 452                                                                                    Cost
## 453                                                                                   Sigma
## 454                                                                                    Cost
## 455                                                                                   Sigma
## 456                                                                                    Cost
## 457                                                                                  Weight
## 458                                                                                  length
## 459                                                                                    Cost
## 460                                                                          Score Function
## 461                                                                     Smoothing Parameter
## 462                                                                                  #Folds
## 463                                                            Minimum Absolute Improvement
## 464                                                                     Smoothing Parameter
## 465                                                               Final Smoothing Parameter
## 466                                                                            Super-Parent
## 467                                                                               parameter
## 468                                                                         Theta Estimated
## 469                                                                         Parallel Curves
## 470                                                                           Link Function
## 471                                                                         Parallel Curves
## 472                                                                           Link Function
## 473                                                                         Parallel Curves
## 474                                                                           Link Function
## 475                                                                             #Components
## 476                                                                            #Fuzzy Terms
## 477                                                                     Membership Function
## 478                                                           #Randomly Selected Predictors
## 479                                                                   # Boosting Iterations
## 480                                                                          Max Tree Depth
## 481                                                                               Shrinkage
## 482                                                                  Minimum Loss Reduction
## 483                                                                    Subsample Percentage
## 484                                                              Subsample Ratio of Columns
## 485                                                               Fraction of Trees Dropped
## 486                                                              Prob. of Skipping Drop-out
## 487                                                          Minimum Sum of Instance Weight
## 488                                                                   # Boosting Iterations
## 489                                                                       L2 Regularization
## 490                                                                       L1 Regularization
## 491                                                                           Learning Rate
## 492                                                                   # Boosting Iterations
## 493                                                                          Max Tree Depth
## 494                                                                               Shrinkage
## 495                                                                  Minimum Loss Reduction
## 496                                                              Subsample Ratio of Columns
## 497                                                          Minimum Sum of Instance Weight
## 498                                                                    Subsample Percentage
## 499                                                                                    Rows
## 500                                                                                 Columns
## 501                                                                            Layer Weight
## 502                                                                                Topology
##     forReg forClass probModel
## 1    FALSE     TRUE      TRUE
## 2    FALSE     TRUE      TRUE
## 3    FALSE     TRUE      TRUE
## 4    FALSE     TRUE      TRUE
## 5    FALSE     TRUE      TRUE
## 9    FALSE     TRUE      TRUE
## 10   FALSE     TRUE      TRUE
## 6    FALSE     TRUE      TRUE
## 7    FALSE     TRUE      TRUE
## 8    FALSE     TRUE      TRUE
## 11   FALSE     TRUE      TRUE
## 12    TRUE    FALSE     FALSE
## 13    TRUE    FALSE     FALSE
## 14    TRUE     TRUE      TRUE
## 15    TRUE     TRUE      TRUE
## 16    TRUE     TRUE      TRUE
## 17   FALSE     TRUE      TRUE
## 18   FALSE     TRUE      TRUE
## 19   FALSE     TRUE      TRUE
## 20    TRUE     TRUE      TRUE
## 21    TRUE     TRUE      TRUE
## 22    TRUE     TRUE      TRUE
## 23    TRUE     TRUE      TRUE
## 24   FALSE     TRUE      TRUE
## 25   FALSE     TRUE      TRUE
## 26   FALSE     TRUE      TRUE
## 27    TRUE     TRUE      TRUE
## 28    TRUE     TRUE      TRUE
## 29    TRUE     TRUE      TRUE
## 30    TRUE     TRUE      TRUE
## 31    TRUE     TRUE      TRUE
## 32    TRUE     TRUE      TRUE
## 33    TRUE     TRUE      TRUE
## 34    TRUE     TRUE      TRUE
## 35   FALSE     TRUE      TRUE
## 36    TRUE     TRUE      TRUE
## 37    TRUE     TRUE      TRUE
## 38    TRUE    FALSE     FALSE
## 39    TRUE    FALSE     FALSE
## 40    TRUE    FALSE     FALSE
## 41    TRUE    FALSE     FALSE
## 42    TRUE     TRUE     FALSE
## 43    TRUE     TRUE     FALSE
## 44    TRUE     TRUE     FALSE
## 45    TRUE     TRUE     FALSE
## 46    TRUE     TRUE     FALSE
## 47    TRUE     TRUE     FALSE
## 48    TRUE     TRUE     FALSE
## 49   FALSE     TRUE      TRUE
## 50   FALSE     TRUE      TRUE
## 51   FALSE     TRUE      TRUE
## 52   FALSE     TRUE     FALSE
## 53   FALSE     TRUE     FALSE
## 54   FALSE     TRUE     FALSE
## 55   FALSE     TRUE     FALSE
## 56   FALSE     TRUE      TRUE
## 57   FALSE     TRUE      TRUE
## 58    TRUE     TRUE      TRUE
## 59   FALSE     TRUE      TRUE
## 60   FALSE     TRUE      TRUE
## 61   FALSE     TRUE      TRUE
## 62   FALSE     TRUE     FALSE
## 63    TRUE     TRUE      TRUE
## 64    TRUE     TRUE      TRUE
## 65    TRUE     TRUE      TRUE
## 66    TRUE    FALSE     FALSE
## 67    TRUE    FALSE     FALSE
## 68   FALSE     TRUE      TRUE
## 69   FALSE     TRUE      TRUE
## 70   FALSE     TRUE     FALSE
## 71   FALSE     TRUE     FALSE
## 72   FALSE     TRUE     FALSE
## 73   FALSE     TRUE     FALSE
## 74   FALSE     TRUE     FALSE
## 75    TRUE    FALSE     FALSE
## 76    TRUE    FALSE     FALSE
## 77    TRUE     TRUE      TRUE
## 78    TRUE     TRUE      TRUE
## 79    TRUE     TRUE      TRUE
## 80    TRUE     TRUE      TRUE
## 81    TRUE     TRUE      TRUE
## 82   FALSE     TRUE      TRUE
## 83   FALSE     TRUE      TRUE
## 84   FALSE     TRUE      TRUE
## 85   FALSE     TRUE      TRUE
## 86   FALSE     TRUE      TRUE
## 87   FALSE     TRUE      TRUE
## 88   FALSE     TRUE      TRUE
## 89   FALSE     TRUE      TRUE
## 90   FALSE     TRUE      TRUE
## 91    TRUE     TRUE      TRUE
## 92    TRUE     TRUE      TRUE
## 93    TRUE     TRUE     FALSE
## 94    TRUE     TRUE     FALSE
## 95    TRUE    FALSE     FALSE
## 96    TRUE    FALSE     FALSE
## 97    TRUE     TRUE      TRUE
## 98    TRUE     TRUE      TRUE
## 99    TRUE     TRUE      TRUE
## 100  FALSE     TRUE      TRUE
## 101  FALSE     TRUE      TRUE
## 102  FALSE     TRUE     FALSE
## 103  FALSE     TRUE     FALSE
## 104  FALSE     TRUE     FALSE
## 105   TRUE    FALSE     FALSE
## 106   TRUE    FALSE     FALSE
## 107   TRUE    FALSE     FALSE
## 108   TRUE    FALSE     FALSE
## 109  FALSE     TRUE     FALSE
## 110  FALSE     TRUE     FALSE
## 111  FALSE     TRUE     FALSE
## 112  FALSE     TRUE     FALSE
## 113   TRUE    FALSE     FALSE
## 114   TRUE    FALSE     FALSE
## 115   TRUE     TRUE      TRUE
## 116   TRUE     TRUE      TRUE
## 117   TRUE     TRUE      TRUE
## 118   TRUE     TRUE      TRUE
## 119   TRUE     TRUE      TRUE
## 120   TRUE     TRUE      TRUE
## 121   TRUE     TRUE      TRUE
## 122   TRUE     TRUE      TRUE
## 123   TRUE     TRUE      TRUE
## 124   TRUE     TRUE      TRUE
## 125   TRUE     TRUE      TRUE
## 131   TRUE     TRUE      TRUE
## 132   TRUE     TRUE      TRUE
## 133   TRUE     TRUE      TRUE
## 134   TRUE     TRUE      TRUE
## 126   TRUE     TRUE      TRUE
## 127   TRUE     TRUE      TRUE
## 128   TRUE     TRUE      TRUE
## 129   TRUE     TRUE      TRUE
## 130   TRUE     TRUE      TRUE
## 135   TRUE     TRUE      TRUE
## 136   TRUE    FALSE     FALSE
## 137   TRUE    FALSE     FALSE
## 138   TRUE    FALSE     FALSE
## 139   TRUE    FALSE     FALSE
## 140   TRUE    FALSE     FALSE
## 141   TRUE    FALSE     FALSE
## 142   TRUE    FALSE     FALSE
## 143   TRUE    FALSE     FALSE
## 144   TRUE    FALSE     FALSE
## 146   TRUE     TRUE      TRUE
## 145   TRUE    FALSE     FALSE
## 147   TRUE     TRUE      TRUE
## 148   TRUE     TRUE      TRUE
## 151   TRUE     TRUE      TRUE
## 152   TRUE     TRUE      TRUE
## 149   TRUE     TRUE      TRUE
## 150   TRUE     TRUE      TRUE
## 153   TRUE     TRUE      TRUE
## 154  FALSE     TRUE      TRUE
## 155  FALSE     TRUE      TRUE
## 156  FALSE     TRUE      TRUE
## 157  FALSE     TRUE      TRUE
## 158  FALSE     TRUE      TRUE
## 159  FALSE     TRUE      TRUE
## 160  FALSE     TRUE      TRUE
## 161  FALSE     TRUE      TRUE
## 162  FALSE     TRUE      TRUE
## 163   TRUE    FALSE     FALSE
## 164   TRUE    FALSE     FALSE
## 165   TRUE    FALSE     FALSE
## 166  FALSE     TRUE      TRUE
## 167  FALSE     TRUE      TRUE
## 168  FALSE     TRUE      TRUE
## 169  FALSE     TRUE      TRUE
## 170  FALSE     TRUE      TRUE
## 171   TRUE     TRUE      TRUE
## 172   TRUE     TRUE      TRUE
## 173   TRUE     TRUE      TRUE
## 174   TRUE     TRUE      TRUE
## 175   TRUE     TRUE      TRUE
## 176   TRUE    FALSE     FALSE
## 177   TRUE    FALSE     FALSE
## 178   TRUE    FALSE     FALSE
## 179   TRUE    FALSE     FALSE
## 180   TRUE    FALSE     FALSE
## 181   TRUE    FALSE     FALSE
## 182   TRUE    FALSE     FALSE
## 183  FALSE     TRUE      TRUE
## 184  FALSE     TRUE      TRUE
## 185   TRUE    FALSE     FALSE
## 186   TRUE    FALSE     FALSE
## 187   TRUE    FALSE     FALSE
## 188  FALSE     TRUE      TRUE
## 189   TRUE    FALSE     FALSE
## 190   TRUE    FALSE     FALSE
## 191  FALSE     TRUE      TRUE
## 192  FALSE     TRUE      TRUE
## 193   TRUE     TRUE      TRUE
## 194   TRUE     TRUE      TRUE
## 195  FALSE     TRUE      TRUE
## 196   TRUE     TRUE      TRUE
## 197   TRUE     TRUE      TRUE
## 198  FALSE     TRUE     FALSE
## 199  FALSE     TRUE     FALSE
## 200  FALSE     TRUE     FALSE
## 201  FALSE     TRUE     FALSE
## 202  FALSE     TRUE     FALSE
## 203  FALSE     TRUE     FALSE
## 204  FALSE     TRUE     FALSE
## 205  FALSE     TRUE     FALSE
## 206   TRUE    FALSE     FALSE
## 207   TRUE    FALSE     FALSE
## 208   TRUE    FALSE     FALSE
## 209   TRUE    FALSE     FALSE
## 210   TRUE    FALSE     FALSE
## 211  FALSE     TRUE      TRUE
## 212  FALSE     TRUE      TRUE
## 213  FALSE     TRUE      TRUE
## 214  FALSE     TRUE     FALSE
## 215   TRUE     TRUE      TRUE
## 216   TRUE     TRUE      TRUE
## 217   TRUE     TRUE      TRUE
## 218   TRUE     TRUE      TRUE
## 219   TRUE     TRUE      TRUE
## 220   TRUE     TRUE      TRUE
## 221   TRUE     TRUE      TRUE
## 222   TRUE     TRUE      TRUE
## 223  FALSE     TRUE      TRUE
## 224  FALSE     TRUE      TRUE
## 225  FALSE     TRUE      TRUE
## 226  FALSE     TRUE      TRUE
## 227  FALSE     TRUE      TRUE
## 228  FALSE     TRUE      TRUE
## 229  FALSE     TRUE      TRUE
## 230  FALSE     TRUE      TRUE
## 231   TRUE     TRUE      TRUE
## 232   TRUE     TRUE      TRUE
## 233   TRUE     TRUE      TRUE
## 234   TRUE     TRUE      TRUE
## 235   TRUE     TRUE      TRUE
## 236   TRUE     TRUE      TRUE
## 237   TRUE     TRUE      TRUE
## 238  FALSE     TRUE      TRUE
## 239  FALSE     TRUE      TRUE
## 240  FALSE     TRUE      TRUE
## 241  FALSE     TRUE      TRUE
## 242  FALSE     TRUE      TRUE
## 243  FALSE     TRUE      TRUE
## 244  FALSE     TRUE      TRUE
## 245  FALSE     TRUE      TRUE
## 246   TRUE     TRUE      TRUE
## 247   TRUE     TRUE      TRUE
## 248   TRUE     TRUE      TRUE
## 249   TRUE     TRUE      TRUE
## 250   TRUE     TRUE      TRUE
## 251   TRUE     TRUE      TRUE
## 252   TRUE     TRUE      TRUE
## 253   TRUE     TRUE      TRUE
## 254   TRUE     TRUE      TRUE
## 255   TRUE     TRUE      TRUE
## 256   TRUE     TRUE      TRUE
## 257   TRUE     TRUE      TRUE
## 258   TRUE     TRUE      TRUE
## 259   TRUE     TRUE      TRUE
## 260   TRUE     TRUE      TRUE
## 261   TRUE     TRUE      TRUE
## 262   TRUE     TRUE      TRUE
## 263   TRUE     TRUE      TRUE
## 264   TRUE     TRUE      TRUE
## 265   TRUE     TRUE      TRUE
## 266   TRUE     TRUE      TRUE
## 267   TRUE     TRUE      TRUE
## 268  FALSE     TRUE      TRUE
## 269   TRUE     TRUE      TRUE
## 270   TRUE     TRUE      TRUE
## 271   TRUE     TRUE      TRUE
## 272   TRUE     TRUE      TRUE
## 273   TRUE     TRUE      TRUE
## 274   TRUE     TRUE      TRUE
## 275   TRUE     TRUE      TRUE
## 276   TRUE     TRUE      TRUE
## 277   TRUE     TRUE      TRUE
## 278   TRUE     TRUE      TRUE
## 279   TRUE     TRUE      TRUE
## 280   TRUE     TRUE      TRUE
## 281   TRUE     TRUE      TRUE
## 282   TRUE     TRUE      TRUE
## 283   TRUE     TRUE      TRUE
## 284  FALSE     TRUE      TRUE
## 285  FALSE     TRUE      TRUE
## 286  FALSE     TRUE      TRUE
## 287  FALSE     TRUE      TRUE
## 288  FALSE     TRUE      TRUE
## 289  FALSE     TRUE      TRUE
## 290  FALSE     TRUE      TRUE
## 291  FALSE     TRUE      TRUE
## 292  FALSE     TRUE      TRUE
## 293  FALSE     TRUE      TRUE
## 294  FALSE     TRUE      TRUE
## 295  FALSE     TRUE      TRUE
## 296   TRUE    FALSE     FALSE
## 297   TRUE    FALSE     FALSE
## 298   TRUE    FALSE     FALSE
## 299   TRUE     TRUE      TRUE
## 300   TRUE     TRUE      TRUE
## 301   TRUE    FALSE     FALSE
## 302   TRUE     TRUE      TRUE
## 303   TRUE     TRUE      TRUE
## 304   TRUE     TRUE      TRUE
## 305  FALSE     TRUE      TRUE
## 306  FALSE     TRUE      TRUE
## 307  FALSE     TRUE      TRUE
## 308  FALSE     TRUE      TRUE
## 309  FALSE     TRUE      TRUE
## 310  FALSE     TRUE      TRUE
## 311  FALSE     TRUE      TRUE
## 312  FALSE     TRUE      TRUE
## 313  FALSE     TRUE      TRUE
## 314  FALSE     TRUE      TRUE
## 315  FALSE     TRUE      TRUE
## 316  FALSE     TRUE     FALSE
## 317  FALSE     TRUE      TRUE
## 318   TRUE     TRUE      TRUE
## 319  FALSE     TRUE      TRUE
## 320  FALSE     TRUE      TRUE
## 321   TRUE     TRUE     FALSE
## 322   TRUE     TRUE     FALSE
## 323   TRUE     TRUE      TRUE
## 324   TRUE     TRUE      TRUE
## 325   TRUE    FALSE     FALSE
## 326  FALSE     TRUE      TRUE
## 327  FALSE     TRUE      TRUE
## 328   TRUE    FALSE     FALSE
## 329   TRUE    FALSE     FALSE
## 330  FALSE     TRUE     FALSE
## 331  FALSE     TRUE     FALSE
## 332  FALSE     TRUE      TRUE
## 333  FALSE     TRUE      TRUE
## 334   TRUE     TRUE      TRUE
## 335   TRUE     TRUE      TRUE
## 336   TRUE     TRUE      TRUE
## 337  FALSE     TRUE      TRUE
## 338   TRUE    FALSE     FALSE
## 339  FALSE     TRUE      TRUE
## 340  FALSE     TRUE      TRUE
## 341  FALSE     TRUE      TRUE
## 342  FALSE     TRUE     FALSE
## 343  FALSE     TRUE     FALSE
## 344  FALSE     TRUE      TRUE
## 345  FALSE     TRUE      TRUE
## 346   TRUE    FALSE     FALSE
## 347   TRUE    FALSE     FALSE
## 348   TRUE    FALSE     FALSE
## 349   TRUE    FALSE     FALSE
## 350   TRUE     TRUE      TRUE
## 351   TRUE     TRUE      TRUE
## 352   TRUE     TRUE      TRUE
## 353   TRUE     TRUE      TRUE
## 354   TRUE     TRUE      TRUE
## 355   TRUE     TRUE      TRUE
## 356   TRUE     TRUE      TRUE
## 357   TRUE     TRUE      TRUE
## 358  FALSE     TRUE      TRUE
## 359  FALSE     TRUE      TRUE
## 360  FALSE     TRUE      TRUE
## 361  FALSE     TRUE      TRUE
## 362  FALSE     TRUE      TRUE
## 363   TRUE    FALSE     FALSE
## 364   TRUE    FALSE     FALSE
## 365   TRUE     TRUE      TRUE
## 366  FALSE     TRUE     FALSE
## 367  FALSE     TRUE     FALSE
## 368   TRUE     TRUE     FALSE
## 369   TRUE     TRUE     FALSE
## 370   TRUE    FALSE     FALSE
## 371  FALSE     TRUE      TRUE
## 372   TRUE    FALSE     FALSE
## 373   TRUE    FALSE     FALSE
## 374  FALSE     TRUE      TRUE
## 375  FALSE     TRUE      TRUE
## 376  FALSE     TRUE     FALSE
## 377  FALSE     TRUE      TRUE
## 378  FALSE     TRUE      TRUE
## 379  FALSE     TRUE      TRUE
## 380  FALSE     TRUE      TRUE
## 381  FALSE     TRUE      TRUE
## 382   TRUE     TRUE      TRUE
## 383   TRUE     TRUE      TRUE
## 384   TRUE     TRUE      TRUE
## 385  FALSE     TRUE     FALSE
## 386  FALSE     TRUE     FALSE
## 387  FALSE     TRUE     FALSE
## 388  FALSE     TRUE     FALSE
## 389  FALSE     TRUE     FALSE
## 390   TRUE    FALSE     FALSE
## 391   TRUE    FALSE     FALSE
## 392   TRUE    FALSE     FALSE
## 393   TRUE     TRUE      TRUE
## 394   TRUE     TRUE      TRUE
## 395   TRUE     TRUE      TRUE
## 396   TRUE     TRUE      TRUE
## 397   TRUE     TRUE      TRUE
## 398  FALSE     TRUE      TRUE
## 399  FALSE     TRUE      TRUE
## 400  FALSE     TRUE      TRUE
## 401  FALSE     TRUE     FALSE
## 402   TRUE    FALSE     FALSE
## 403   TRUE    FALSE     FALSE
## 404   TRUE    FALSE     FALSE
## 405   TRUE    FALSE     FALSE
## 406   TRUE    FALSE     FALSE
## 407   TRUE    FALSE     FALSE
## 408   TRUE    FALSE     FALSE
## 409  FALSE     TRUE      TRUE
## 410  FALSE     TRUE      TRUE
## 411  FALSE     TRUE      TRUE
## 412  FALSE     TRUE      TRUE
## 413   TRUE     TRUE      TRUE
## 414  FALSE     TRUE     FALSE
## 415  FALSE     TRUE     FALSE
## 416  FALSE     TRUE     FALSE
## 417  FALSE     TRUE      TRUE
## 418  FALSE     TRUE     FALSE
## 419  FALSE     TRUE     FALSE
## 420  FALSE     TRUE     FALSE
## 421  FALSE     TRUE     FALSE
## 422  FALSE     TRUE      TRUE
## 423  FALSE     TRUE      TRUE
## 424   TRUE    FALSE     FALSE
## 425   TRUE     TRUE      TRUE
## 426   TRUE     TRUE      TRUE
## 427   TRUE     TRUE      TRUE
## 428  FALSE     TRUE      TRUE
## 429  FALSE     TRUE      TRUE
## 430  FALSE     TRUE      TRUE
## 431  FALSE     TRUE      TRUE
## 432   TRUE    FALSE     FALSE
## 433   TRUE    FALSE     FALSE
## 434   TRUE     TRUE      TRUE
## 435   TRUE     TRUE      TRUE
## 436   TRUE     TRUE      TRUE
## 437   TRUE     TRUE      TRUE
## 438   TRUE     TRUE      TRUE
## 439   TRUE     TRUE      TRUE
## 440   TRUE     TRUE     FALSE
## 441   TRUE     TRUE     FALSE
## 442  FALSE     TRUE      TRUE
## 443  FALSE     TRUE      TRUE
## 444  FALSE     TRUE     FALSE
## 445  FALSE     TRUE     FALSE
## 446  FALSE     TRUE     FALSE
## 447   TRUE     TRUE      TRUE
## 448   TRUE     TRUE      TRUE
## 449   TRUE     TRUE      TRUE
## 450   TRUE     TRUE      TRUE
## 451   TRUE     TRUE      TRUE
## 452   TRUE     TRUE      TRUE
## 453   TRUE     TRUE      TRUE
## 454   TRUE     TRUE      TRUE
## 455  FALSE     TRUE      TRUE
## 456  FALSE     TRUE      TRUE
## 457  FALSE     TRUE      TRUE
## 458   TRUE     TRUE      TRUE
## 459   TRUE     TRUE      TRUE
## 460  FALSE     TRUE      TRUE
## 461  FALSE     TRUE      TRUE
## 462  FALSE     TRUE      TRUE
## 463  FALSE     TRUE      TRUE
## 464  FALSE     TRUE      TRUE
## 465  FALSE     TRUE      TRUE
## 466  FALSE     TRUE      TRUE
## 467   TRUE     TRUE      TRUE
## 468  FALSE     TRUE      TRUE
## 469  FALSE     TRUE      TRUE
## 470  FALSE     TRUE      TRUE
## 471  FALSE     TRUE      TRUE
## 472  FALSE     TRUE      TRUE
## 473  FALSE     TRUE      TRUE
## 474  FALSE     TRUE      TRUE
## 475   TRUE     TRUE      TRUE
## 476   TRUE    FALSE     FALSE
## 477   TRUE    FALSE     FALSE
## 478  FALSE     TRUE      TRUE
## 479   TRUE     TRUE      TRUE
## 480   TRUE     TRUE      TRUE
## 481   TRUE     TRUE      TRUE
## 482   TRUE     TRUE      TRUE
## 483   TRUE     TRUE      TRUE
## 484   TRUE     TRUE      TRUE
## 485   TRUE     TRUE      TRUE
## 486   TRUE     TRUE      TRUE
## 487   TRUE     TRUE      TRUE
## 488   TRUE     TRUE      TRUE
## 489   TRUE     TRUE      TRUE
## 490   TRUE     TRUE      TRUE
## 491   TRUE     TRUE      TRUE
## 492   TRUE     TRUE      TRUE
## 493   TRUE     TRUE      TRUE
## 494   TRUE     TRUE      TRUE
## 495   TRUE     TRUE      TRUE
## 496   TRUE     TRUE      TRUE
## 497   TRUE     TRUE      TRUE
## 498   TRUE     TRUE      TRUE
## 499   TRUE     TRUE      TRUE
## 500   TRUE     TRUE      TRUE
## 501   TRUE     TRUE      TRUE
## 502   TRUE     TRUE      TRUE
\end{verbatim}

\begin{Shaded}
\begin{Highlighting}[]
\NormalTok{trainIndex }\OtherTok{\textless{}{-}} \FunctionTok{createDataPartition}\NormalTok{(dfR}\SpecialCharTok{$}\NormalTok{BRAIN\_VOLUME, }
                                  \AttributeTok{p =}\NormalTok{ .}\DecValTok{8}\NormalTok{, }
                                  \AttributeTok{list =} \ConstantTok{FALSE}\NormalTok{, }
                                  \AttributeTok{times =} \DecValTok{1}\NormalTok{)}

\NormalTok{b\_train }\OtherTok{\textless{}{-}}\NormalTok{ dfR[trainIndex, ]}
\end{Highlighting}
\end{Shaded}

\begin{verbatim}
## Warning: The `i` argument of ``[`()` can't be a matrix as of tibble 3.0.0.
## Convert to a vector.
## This warning is displayed once every 8 hours.
## Call `lifecycle::last_warnings()` to see where this warning was generated.
\end{verbatim}

\begin{Shaded}
\begin{Highlighting}[]
\NormalTok{b\_test  }\OtherTok{\textless{}{-}}\NormalTok{ dfR[}\SpecialCharTok{{-}}\NormalTok{trainIndex, ]}
\end{Highlighting}
\end{Shaded}

\begin{Shaded}
\begin{Highlighting}[]
\NormalTok{fitControl }\OtherTok{\textless{}{-}} \FunctionTok{trainControl}\NormalTok{(}\AttributeTok{method =} \StringTok{"repeatedcv"}\NormalTok{,}
                           \AttributeTok{number =} \DecValTok{10}\NormalTok{,}
                           \AttributeTok{repeats =} \DecValTok{1}\NormalTok{)}

\CommentTok{\# Y esta función entrena el método}
\NormalTok{caret\_lm\_model }\OtherTok{\textless{}{-}} \FunctionTok{train}\NormalTok{(BRAIN\_VOLUME }\SpecialCharTok{\textasciitilde{}}\NormalTok{ ., }
                        \AttributeTok{data =}\NormalTok{ b\_train, }
                        \AttributeTok{method =} \StringTok{"lm"}\NormalTok{,}
                        \AttributeTok{trControl =}\NormalTok{ fitControl)}
\end{Highlighting}
\end{Shaded}

\begin{Shaded}
\begin{Highlighting}[]
\NormalTok{caret\_lm\_model}
\end{Highlighting}
\end{Shaded}

\begin{verbatim}
## Linear Regression 
## 
## 210 samples
##   2 predictor
## 
## No pre-processing
## Resampling: Cross-Validated (10 fold, repeated 1 times) 
## Summary of sample sizes: 190, 188, 189, 189, 188, 190, ... 
## Resampling results:
## 
##   RMSE      Rsquared   MAE     
##   109.7582  0.4764145  84.86985
## 
## Tuning parameter 'intercept' was held constant at a value of TRUE
\end{verbatim}

Podemos observar que se ha conseguido un MSE = 11224.32.

Comparación entre modelos.

Con esto, somos capaces de seleccionar el mejor de los métodos (el que
minimice el error cometido):

\begin{Shaded}
\begin{Highlighting}[]
\CommentTok{\#install.packages("kableExtra", dependencies = TRUE)}

\CommentTok{\# Recorremos todos los métodos y tomamos el RMSE de cada uno}
\NormalTok{best\_rmse }\OtherTok{\textless{}{-}} \FunctionTok{sapply}\NormalTok{(caret\_results, }\ControlFlowTok{function}\NormalTok{(i) }\FunctionTok{min}\NormalTok{(i}\SpecialCharTok{$}\NormalTok{results}\SpecialCharTok{$}\NormalTok{RMSE, }\AttributeTok{na.rm =} \ConstantTok{TRUE}\NormalTok{))}

\CommentTok{\# El mejor será el que tenga un RMSE más bajo}
\NormalTok{best\_regressor }\OtherTok{\textless{}{-}} \FunctionTok{names}\NormalTok{(best\_rmse)[}\FunctionTok{which.min}\NormalTok{(best\_rmse)]}

\CommentTok{\# Los mostramos en forma de tabla}
\NormalTok{w }\OtherTok{\textless{}{-}}\NormalTok{ best\_rmse }\SpecialCharTok{\%\textgreater{}\%} \FunctionTok{as.data.frame}\NormalTok{()}

\FunctionTok{colnames}\NormalTok{(w) }\OtherTok{\textless{}{-}} \StringTok{"RMSE"}

\NormalTok{w }\SpecialCharTok{\%\textgreater{}\%} 
\NormalTok{  knitr}\SpecialCharTok{::}\FunctionTok{kable}\NormalTok{(}\AttributeTok{format =} \StringTok{"html"}\NormalTok{) }\SpecialCharTok{\%\textgreater{}\%}\NormalTok{ kableExtra}\SpecialCharTok{::}\FunctionTok{kable\_styling}\NormalTok{(}\AttributeTok{font\_size =} \DecValTok{14}\NormalTok{)}
\end{Highlighting}
\end{Shaded}

RMSE

knn

124.0474

gbm

110.8667

lm

110.2659

\begin{Shaded}
\begin{Highlighting}[]
\FunctionTok{cat}\NormalTok{(}\StringTok{"El metodo que minimiza la medida del error:"}\NormalTok{,best\_regressor)}
\end{Highlighting}
\end{Shaded}

\begin{verbatim}
## El metodo que minimiza la medida del error: lm
\end{verbatim}

Visualización de resultados.

\begin{Shaded}
\begin{Highlighting}[]
\FunctionTok{plot}\NormalTok{(caret\_results}\SpecialCharTok{$}\NormalTok{gbm)}
\end{Highlighting}
\end{Shaded}

\includegraphics{ProyectoContrastesRegresion_files/figure-latex/unnamed-chunk-34-1.pdf}

Vamos a repetir los 2 pasos previos para encontrar modelos que estimen
la variable GM\_VOLUME (volumen de sustancia gris) y modelos para
estimar WM\_VOLUME (volumen de sustancia blanca) a partir de la edad y
del sexo.

\begin{Shaded}
\begin{Highlighting}[]
\NormalTok{dfRG}\OtherTok{=}\NormalTok{ df2 }\SpecialCharTok{\%\textgreater{}\%} \FunctionTok{select}\NormalTok{(SEX,AGE,GM\_VOLUME)}

\NormalTok{dfRW}\OtherTok{=}\NormalTok{ df2 }\SpecialCharTok{\%\textgreater{}\%} \FunctionTok{select}\NormalTok{(SEX,AGE,WM\_VOLUME)}
\end{Highlighting}
\end{Shaded}

Modelo de regresion simple y polinomico (volumen de sustancia gris).

\begin{Shaded}
\begin{Highlighting}[]
\NormalTok{modelo\_1G }\OtherTok{\textless{}{-}} \FunctionTok{lm}\NormalTok{(GM\_VOLUME }\SpecialCharTok{\textasciitilde{}}\NormalTok{ AGE, }\AttributeTok{data =}\NormalTok{ dfRG)}
\NormalTok{modelo\_2G }\OtherTok{\textless{}{-}} \FunctionTok{lm}\NormalTok{(GM\_VOLUME }\SpecialCharTok{\textasciitilde{}} \FunctionTok{poly}\NormalTok{(AGE, }\DecValTok{2}\NormalTok{), }\AttributeTok{data =}\NormalTok{ dfRG)}
\NormalTok{modelo\_3G }\OtherTok{\textless{}{-}} \FunctionTok{lm}\NormalTok{(GM\_VOLUME }\SpecialCharTok{\textasciitilde{}} \FunctionTok{poly}\NormalTok{(AGE, }\DecValTok{3}\NormalTok{), }\AttributeTok{data =}\NormalTok{ dfRG)}
\NormalTok{modelo\_4G }\OtherTok{\textless{}{-}} \FunctionTok{lm}\NormalTok{(GM\_VOLUME }\SpecialCharTok{\textasciitilde{}} \FunctionTok{poly}\NormalTok{(AGE, }\DecValTok{4}\NormalTok{), }\AttributeTok{data =}\NormalTok{ dfRG)}
\NormalTok{modelo\_5G }\OtherTok{\textless{}{-}} \FunctionTok{lm}\NormalTok{(GM\_VOLUME }\SpecialCharTok{\textasciitilde{}} \FunctionTok{poly}\NormalTok{(AGE, }\DecValTok{5}\NormalTok{), }\AttributeTok{data =}\NormalTok{ dfRG)}

\FunctionTok{anova}\NormalTok{(modelo\_1G, modelo\_2G, modelo\_3G, modelo\_4G, modelo\_5G)}
\end{Highlighting}
\end{Shaded}

\begin{verbatim}
## Analysis of Variance Table
## 
## Model 1: GM_VOLUME ~ AGE
## Model 2: GM_VOLUME ~ poly(AGE, 2)
## Model 3: GM_VOLUME ~ poly(AGE, 3)
## Model 4: GM_VOLUME ~ poly(AGE, 4)
## Model 5: GM_VOLUME ~ poly(AGE, 5)
##   Res.Df     RSS Df Sum of Sq       F   Pr(>F)   
## 1    260 1111728                                 
## 2    259 1068505  1     43223 10.5477 0.001319 **
## 3    258 1068369  1       136  0.0331 0.855815   
## 4    257 1049153  1     19216  4.6894 0.031273 * 
## 5    256 1049043  1       110  0.0268 0.870076   
## ---
## Signif. codes:  0 '***' 0.001 '**' 0.01 '*' 0.05 '.' 0.1 ' ' 1
\end{verbatim}

\begin{Shaded}
\begin{Highlighting}[]
\NormalTok{f1G}\OtherTok{=} \FunctionTok{lm}\NormalTok{(dfRG}\SpecialCharTok{$}\NormalTok{GM\_VOLUME }\SpecialCharTok{\textasciitilde{}}\NormalTok{ dfRG}\SpecialCharTok{$}\NormalTok{AGE }\SpecialCharTok{+}\NormalTok{ dfRG}\SpecialCharTok{$}\NormalTok{SEX)}
\NormalTok{f2G}\OtherTok{=} \FunctionTok{lm}\NormalTok{(dfRG}\SpecialCharTok{$}\NormalTok{GM\_VOLUME }\SpecialCharTok{\textasciitilde{}}\NormalTok{ dfRG}\SpecialCharTok{$}\NormalTok{AGE }\SpecialCharTok{+}\NormalTok{ dfRG}\SpecialCharTok{$}\NormalTok{SEX }\SpecialCharTok{+} \FunctionTok{I}\NormalTok{(dfRG}\SpecialCharTok{$}\NormalTok{AGE}\SpecialCharTok{\^{}}\DecValTok{2}\NormalTok{))}
\NormalTok{f3G}\OtherTok{=} \FunctionTok{lm}\NormalTok{(dfRG}\SpecialCharTok{$}\NormalTok{GM\_VOLUME }\SpecialCharTok{\textasciitilde{}}\NormalTok{ dfRG}\SpecialCharTok{$}\NormalTok{AGE }\SpecialCharTok{+}\NormalTok{ dfRG}\SpecialCharTok{$}\NormalTok{SEX }\SpecialCharTok{+} \FunctionTok{I}\NormalTok{(dfRG}\SpecialCharTok{$}\NormalTok{AGE}\SpecialCharTok{\^{}}\DecValTok{2}\NormalTok{) }\SpecialCharTok{+} \FunctionTok{I}\NormalTok{(dfRG}\SpecialCharTok{$}\NormalTok{AGE}\SpecialCharTok{\^{}}\DecValTok{3}\NormalTok{))}
\NormalTok{f4G}\OtherTok{=} \FunctionTok{lm}\NormalTok{(dfRG}\SpecialCharTok{$}\NormalTok{GM\_VOLUME }\SpecialCharTok{\textasciitilde{}}\NormalTok{ dfRG}\SpecialCharTok{$}\NormalTok{AGE }\SpecialCharTok{+}\NormalTok{ dfRG}\SpecialCharTok{$}\NormalTok{SEX }\SpecialCharTok{+} \FunctionTok{I}\NormalTok{(dfRG}\SpecialCharTok{$}\NormalTok{AGE}\SpecialCharTok{\^{}}\DecValTok{2}\NormalTok{) }\SpecialCharTok{+} \FunctionTok{I}\NormalTok{(dfRG}\SpecialCharTok{$}\NormalTok{AGE}\SpecialCharTok{\^{}}\DecValTok{3}\NormalTok{) }\SpecialCharTok{+} \FunctionTok{I}\NormalTok{(dfRG}\SpecialCharTok{$}\NormalTok{AGE}\SpecialCharTok{\^{}}\DecValTok{4}\NormalTok{))}
\NormalTok{f5G}\OtherTok{=} \FunctionTok{lm}\NormalTok{(dfRG}\SpecialCharTok{$}\NormalTok{GM\_VOLUME }\SpecialCharTok{\textasciitilde{}}\NormalTok{ dfRG}\SpecialCharTok{$}\NormalTok{AGE }\SpecialCharTok{+}\NormalTok{ dfRG}\SpecialCharTok{$}\NormalTok{SEX }\SpecialCharTok{+} \FunctionTok{I}\NormalTok{(dfRG}\SpecialCharTok{$}\NormalTok{AGE}\SpecialCharTok{\^{}}\DecValTok{2}\NormalTok{) }\SpecialCharTok{+} \FunctionTok{I}\NormalTok{(dfRG}\SpecialCharTok{$}\NormalTok{AGE}\SpecialCharTok{\^{}}\DecValTok{3}\NormalTok{) }\SpecialCharTok{+} \FunctionTok{I}\NormalTok{(dfRG}\SpecialCharTok{$}\NormalTok{AGE}\SpecialCharTok{\^{}}\DecValTok{4}\NormalTok{) }\SpecialCharTok{+} \FunctionTok{I}\NormalTok{(dfRG}\SpecialCharTok{$}\NormalTok{AGE}\SpecialCharTok{\^{}}\DecValTok{5}\NormalTok{))}
\FunctionTok{anova}\NormalTok{(f1G,f2G,f3G,f4G,f5G)}
\end{Highlighting}
\end{Shaded}

\begin{verbatim}
## Analysis of Variance Table
## 
## Model 1: dfRG$GM_VOLUME ~ dfRG$AGE + dfRG$SEX
## Model 2: dfRG$GM_VOLUME ~ dfRG$AGE + dfRG$SEX + I(dfRG$AGE^2)
## Model 3: dfRG$GM_VOLUME ~ dfRG$AGE + dfRG$SEX + I(dfRG$AGE^2) + I(dfRG$AGE^3)
## Model 4: dfRG$GM_VOLUME ~ dfRG$AGE + dfRG$SEX + I(dfRG$AGE^2) + I(dfRG$AGE^3) + 
##     I(dfRG$AGE^4)
## Model 5: dfRG$GM_VOLUME ~ dfRG$AGE + dfRG$SEX + I(dfRG$AGE^2) + I(dfRG$AGE^3) + 
##     I(dfRG$AGE^4) + I(dfRG$AGE^5)
##   Res.Df    RSS Df Sum of Sq      F  Pr(>F)  
## 1    259 837304                              
## 2    258 816401  1   20902.9 6.7248 0.01006 *
## 3    257 812178  1    4223.3 1.3587 0.24485  
## 4    256 792634  1   19544.2 6.2877 0.01278 *
## 5    255 792620  1      14.0 0.0045 0.94663  
## ---
## Signif. codes:  0 '***' 0.001 '**' 0.01 '*' 0.05 '.' 0.1 ' ' 1
\end{verbatim}

\#degree 2 is the better

\begin{Shaded}
\begin{Highlighting}[]
\FunctionTok{summary}\NormalTok{(f2G) }
\end{Highlighting}
\end{Shaded}

\begin{verbatim}
## 
## Call:
## lm(formula = dfRG$GM_VOLUME ~ dfRG$AGE + dfRG$SEX + I(dfRG$AGE^2))
## 
## Residuals:
##      Min       1Q   Median       3Q      Max 
## -194.297  -32.849    1.286   32.477  170.276 
## 
## Coefficients:
##                Estimate Std. Error t value Pr(>|t|)    
## (Intercept)   149.60912  223.41050   0.670   0.5037    
## dfRG$AGE       13.88024    6.04425   2.296   0.0225 *  
## dfRG$SEXMALE   63.50071    7.11428   8.926   <2e-16 ***
## I(dfRG$AGE^2)  -0.10447    0.04065  -2.570   0.0107 *  
## ---
## Signif. codes:  0 '***' 0.001 '**' 0.01 '*' 0.05 '.' 0.1 ' ' 1
## 
## Residual standard error: 56.25 on 258 degrees of freedom
## Multiple R-squared:  0.2907, Adjusted R-squared:  0.2824 
## F-statistic: 35.24 on 3 and 258 DF,  p-value: < 2.2e-16
\end{verbatim}

p-value: \textless{} 2.2e-16 por lo tanto se acepta el modelo. En el
caso del predictor AGE, si el resto de variables no varían, por cada
unidad de AGE que aumenta el volumen cerebral se aumenta en promedio
13.88024 unidades. Las variables son significativas (al menos `*' 0.05),
Coeficiente de determinacion R\^{}2(0.2907 o 29\%): es el porcentaje de
la variación en la variable de respuesta que es explicado por un modelo
lineal.

\begin{Shaded}
\begin{Highlighting}[]
\FunctionTok{plot.new}\NormalTok{()}
\FunctionTok{plot}\NormalTok{(}\FunctionTok{fitted}\NormalTok{(f2G), }\FunctionTok{residuals}\NormalTok{(f2G), }\AttributeTok{xlab =} \StringTok{"Approached values"}\NormalTok{,  }\AttributeTok{ylab =} \StringTok{"Residuals"}\NormalTok{)}
  \FunctionTok{abline}\NormalTok{(}\AttributeTok{h=}\DecValTok{0}\NormalTok{, }\AttributeTok{lty=}\DecValTok{2}\NormalTok{)}
    \FunctionTok{lines}\NormalTok{(}\FunctionTok{smooth.spline}\NormalTok{(}\FunctionTok{fitted}\NormalTok{(f2G), }\FunctionTok{residuals}\NormalTok{(f2G)))}
\end{Highlighting}
\end{Shaded}

\includegraphics{ProyectoContrastesRegresion_files/figure-latex/unnamed-chunk-39-1.pdf}

\begin{Shaded}
\begin{Highlighting}[]
\FunctionTok{plot}\NormalTok{(f2G}\SpecialCharTok{$}\NormalTok{residuals)}
\end{Highlighting}
\end{Shaded}

\includegraphics{ProyectoContrastesRegresion_files/figure-latex/unnamed-chunk-40-1.pdf}

\begin{Shaded}
\begin{Highlighting}[]
\FunctionTok{summary}\NormalTok{(f2G}\SpecialCharTok{$}\NormalTok{residuals)}
\end{Highlighting}
\end{Shaded}

\begin{verbatim}
##     Min.  1st Qu.   Median     Mean  3rd Qu.     Max. 
## -194.297  -32.849    1.286    0.000   32.477  170.275
\end{verbatim}

\begin{Shaded}
\begin{Highlighting}[]
\FunctionTok{boxplot}\NormalTok{(f2G}\SpecialCharTok{$}\NormalTok{residuals)}
\end{Highlighting}
\end{Shaded}

\includegraphics{ProyectoContrastesRegresion_files/figure-latex/unnamed-chunk-40-2.pdf}
CARET para modelos del volumen de sustancia gris.

\begin{Shaded}
\begin{Highlighting}[]
\NormalTok{trainIndex }\OtherTok{\textless{}{-}} \FunctionTok{createDataPartition}\NormalTok{(dfRG}\SpecialCharTok{$}\NormalTok{GM\_VOLUME, }
                                  \AttributeTok{p =}\NormalTok{ .}\DecValTok{8}\NormalTok{, }
                                  \AttributeTok{list =} \ConstantTok{FALSE}\NormalTok{, }
                                  \AttributeTok{times =} \DecValTok{1}\NormalTok{)}

\NormalTok{b\_train }\OtherTok{\textless{}{-}}\NormalTok{ dfRG[trainIndex, ]}
\NormalTok{b\_test  }\OtherTok{\textless{}{-}}\NormalTok{ dfRG[}\SpecialCharTok{{-}}\NormalTok{trainIndex, ]}
\end{Highlighting}
\end{Shaded}

\begin{Shaded}
\begin{Highlighting}[]
\NormalTok{fitControl }\OtherTok{\textless{}{-}} \FunctionTok{trainControl}\NormalTok{(}\AttributeTok{method =} \StringTok{"repeatedcv"}\NormalTok{,}
                           \AttributeTok{number =} \DecValTok{10}\NormalTok{,}
                           \AttributeTok{repeats =} \DecValTok{1}\NormalTok{)}

\CommentTok{\# Y esta función entrena el método}
\NormalTok{caret\_lm\_model }\OtherTok{\textless{}{-}} \FunctionTok{train}\NormalTok{(GM\_VOLUME }\SpecialCharTok{\textasciitilde{}}\NormalTok{ ., }
                        \AttributeTok{data =}\NormalTok{ b\_train, }
                        \AttributeTok{method =} \StringTok{"lm"}\NormalTok{,}
                        \AttributeTok{trControl =}\NormalTok{ fitControl)}
\end{Highlighting}
\end{Shaded}

\begin{Shaded}
\begin{Highlighting}[]
\NormalTok{caret\_lm\_model}
\end{Highlighting}
\end{Shaded}

\begin{verbatim}
## Linear Regression 
## 
## 211 samples
##   2 predictor
## 
## No pre-processing
## Resampling: Cross-Validated (10 fold, repeated 1 times) 
## Summary of sample sizes: 191, 189, 191, 189, 190, 191, ... 
## Resampling results:
## 
##   RMSE      Rsquared   MAE     
##   53.67309  0.2873145  42.39242
## 
## Tuning parameter 'intercept' was held constant at a value of TRUE
\end{verbatim}

Podemos observar que se ha conseguido un MSE = 3155.691

Comparación entre modelos.

Con esto, somos capaces de seleccionar el mejor de los métodos (el que
minimice el error cometido):

\begin{Shaded}
\begin{Highlighting}[]
\CommentTok{\#install.packages("kableExtra", dependencies = TRUE)}

\CommentTok{\# Recorremos todos los métodos y tomamos el RMSE de cada uno}
\NormalTok{best\_rmse }\OtherTok{\textless{}{-}} \FunctionTok{sapply}\NormalTok{(caret\_results, }\ControlFlowTok{function}\NormalTok{(i) }\FunctionTok{min}\NormalTok{(i}\SpecialCharTok{$}\NormalTok{results}\SpecialCharTok{$}\NormalTok{RMSE, }\AttributeTok{na.rm =} \ConstantTok{TRUE}\NormalTok{))}

\CommentTok{\# El mejor será el que tenga un RMSE más bajo}
\NormalTok{best\_regressor }\OtherTok{\textless{}{-}} \FunctionTok{names}\NormalTok{(best\_rmse)[}\FunctionTok{which.min}\NormalTok{(best\_rmse)]}

\CommentTok{\# Los mostramos en forma de tabla}
\NormalTok{w }\OtherTok{\textless{}{-}}\NormalTok{ best\_rmse }\SpecialCharTok{\%\textgreater{}\%} \FunctionTok{as.data.frame}\NormalTok{()}

\FunctionTok{colnames}\NormalTok{(w) }\OtherTok{\textless{}{-}} \StringTok{"RMSE"}

\NormalTok{w }\SpecialCharTok{\%\textgreater{}\%} 
\NormalTok{  knitr}\SpecialCharTok{::}\FunctionTok{kable}\NormalTok{(}\AttributeTok{format =} \StringTok{"html"}\NormalTok{) }\SpecialCharTok{\%\textgreater{}\%}\NormalTok{ kableExtra}\SpecialCharTok{::}\FunctionTok{kable\_styling}\NormalTok{(}\AttributeTok{font\_size =} \DecValTok{14}\NormalTok{)}
\end{Highlighting}
\end{Shaded}

RMSE

knn

58.03122

gbm

51.30519

lm

53.33848

\begin{Shaded}
\begin{Highlighting}[]
\FunctionTok{cat}\NormalTok{(}\StringTok{"El metodo que minimiza la medida del error:"}\NormalTok{,best\_regressor)}
\end{Highlighting}
\end{Shaded}

\begin{verbatim}
## El metodo que minimiza la medida del error: gbm
\end{verbatim}

Visualización de resultados.

\begin{Shaded}
\begin{Highlighting}[]
\FunctionTok{plot}\NormalTok{(caret\_results}\SpecialCharTok{$}\NormalTok{gbm)}
\end{Highlighting}
\end{Shaded}

\includegraphics{ProyectoContrastesRegresion_files/figure-latex/unnamed-chunk-47-1.pdf}

modelo de regresion simple y polinomico para modelos con volumen de
sustancia blanca.

\begin{Shaded}
\begin{Highlighting}[]
\NormalTok{modelo\_1W }\OtherTok{\textless{}{-}} \FunctionTok{lm}\NormalTok{(WM\_VOLUME }\SpecialCharTok{\textasciitilde{}}\NormalTok{ AGE, }\AttributeTok{data =}\NormalTok{ dfRW)}
\NormalTok{modelo\_2W }\OtherTok{\textless{}{-}} \FunctionTok{lm}\NormalTok{(WM\_VOLUME }\SpecialCharTok{\textasciitilde{}} \FunctionTok{poly}\NormalTok{(AGE, }\DecValTok{2}\NormalTok{), }\AttributeTok{data =}\NormalTok{ dfRW)}
\NormalTok{modelo\_3W }\OtherTok{\textless{}{-}} \FunctionTok{lm}\NormalTok{(WM\_VOLUME }\SpecialCharTok{\textasciitilde{}} \FunctionTok{poly}\NormalTok{(AGE, }\DecValTok{3}\NormalTok{), }\AttributeTok{data =}\NormalTok{ dfRW)}
\NormalTok{modelo\_4W }\OtherTok{\textless{}{-}} \FunctionTok{lm}\NormalTok{(WM\_VOLUME }\SpecialCharTok{\textasciitilde{}} \FunctionTok{poly}\NormalTok{(AGE, }\DecValTok{4}\NormalTok{), }\AttributeTok{data =}\NormalTok{ dfRW)}
\NormalTok{modelo\_5W }\OtherTok{\textless{}{-}} \FunctionTok{lm}\NormalTok{(WM\_VOLUME }\SpecialCharTok{\textasciitilde{}} \FunctionTok{poly}\NormalTok{(AGE, }\DecValTok{5}\NormalTok{), }\AttributeTok{data =}\NormalTok{ dfRW)}

\FunctionTok{anova}\NormalTok{(modelo\_1W, modelo\_2W, modelo\_3W, modelo\_4W, modelo\_5W)}
\end{Highlighting}
\end{Shaded}

\begin{verbatim}
## Analysis of Variance Table
## 
## Model 1: WM_VOLUME ~ AGE
## Model 2: WM_VOLUME ~ poly(AGE, 2)
## Model 3: WM_VOLUME ~ poly(AGE, 3)
## Model 4: WM_VOLUME ~ poly(AGE, 4)
## Model 5: WM_VOLUME ~ poly(AGE, 5)
##   Res.Df    RSS Df Sum of Sq      F  Pr(>F)  
## 1    260 801032                              
## 2    259 787862  1   13170.2 4.3057 0.03898 *
## 3    258 787248  1     613.1 0.2004 0.65475  
## 4    257 783579  1    3669.5 1.1997 0.27442  
## 5    256 783042  1     537.4 0.1757 0.67545  
## ---
## Signif. codes:  0 '***' 0.001 '**' 0.01 '*' 0.05 '.' 0.1 ' ' 1
\end{verbatim}

\begin{Shaded}
\begin{Highlighting}[]
\NormalTok{f1W}\OtherTok{=} \FunctionTok{lm}\NormalTok{(dfRW}\SpecialCharTok{$}\NormalTok{WM\_VOLUME }\SpecialCharTok{\textasciitilde{}}\NormalTok{ dfRW}\SpecialCharTok{$}\NormalTok{AGE }\SpecialCharTok{+}\NormalTok{ dfRW}\SpecialCharTok{$}\NormalTok{SEX)}
\NormalTok{f2W}\OtherTok{=} \FunctionTok{lm}\NormalTok{(dfRW}\SpecialCharTok{$}\NormalTok{WM\_VOLUME }\SpecialCharTok{\textasciitilde{}}\NormalTok{ dfRW}\SpecialCharTok{$}\NormalTok{AGE }\SpecialCharTok{+}\NormalTok{ dfRW}\SpecialCharTok{$}\NormalTok{SEX }\SpecialCharTok{+} \FunctionTok{I}\NormalTok{(dfRW}\SpecialCharTok{$}\NormalTok{AGE}\SpecialCharTok{\^{}}\DecValTok{2}\NormalTok{))}
\NormalTok{f3W}\OtherTok{=} \FunctionTok{lm}\NormalTok{(dfRW}\SpecialCharTok{$}\NormalTok{WM\_VOLUME }\SpecialCharTok{\textasciitilde{}}\NormalTok{ dfRW}\SpecialCharTok{$}\NormalTok{AGE }\SpecialCharTok{+}\NormalTok{ dfRW}\SpecialCharTok{$}\NormalTok{SEX }\SpecialCharTok{+} \FunctionTok{I}\NormalTok{(dfRW}\SpecialCharTok{$}\NormalTok{AGE}\SpecialCharTok{\^{}}\DecValTok{2}\NormalTok{) }\SpecialCharTok{+} \FunctionTok{I}\NormalTok{(dfRW}\SpecialCharTok{$}\NormalTok{AGE}\SpecialCharTok{\^{}}\DecValTok{3}\NormalTok{))}
\NormalTok{f4W}\OtherTok{=} \FunctionTok{lm}\NormalTok{(dfRW}\SpecialCharTok{$}\NormalTok{WM\_VOLUME }\SpecialCharTok{\textasciitilde{}}\NormalTok{ dfRW}\SpecialCharTok{$}\NormalTok{AGE }\SpecialCharTok{+}\NormalTok{ dfRW}\SpecialCharTok{$}\NormalTok{SEX }\SpecialCharTok{+} \FunctionTok{I}\NormalTok{(dfRW}\SpecialCharTok{$}\NormalTok{AGE}\SpecialCharTok{\^{}}\DecValTok{2}\NormalTok{) }\SpecialCharTok{+} \FunctionTok{I}\NormalTok{(dfRW}\SpecialCharTok{$}\NormalTok{AGE}\SpecialCharTok{\^{}}\DecValTok{3}\NormalTok{) }\SpecialCharTok{+} \FunctionTok{I}\NormalTok{(dfRW}\SpecialCharTok{$}\NormalTok{AGE}\SpecialCharTok{\^{}}\DecValTok{4}\NormalTok{))}
\NormalTok{f5W}\OtherTok{=} \FunctionTok{lm}\NormalTok{(dfRW}\SpecialCharTok{$}\NormalTok{WM\_VOLUME }\SpecialCharTok{\textasciitilde{}}\NormalTok{ dfRW}\SpecialCharTok{$}\NormalTok{AGE }\SpecialCharTok{+}\NormalTok{ dfRW}\SpecialCharTok{$}\NormalTok{SEX }\SpecialCharTok{+} \FunctionTok{I}\NormalTok{(dfRW}\SpecialCharTok{$}\NormalTok{AGE}\SpecialCharTok{\^{}}\DecValTok{2}\NormalTok{) }\SpecialCharTok{+} \FunctionTok{I}\NormalTok{(dfRW}\SpecialCharTok{$}\NormalTok{AGE}\SpecialCharTok{\^{}}\DecValTok{3}\NormalTok{) }\SpecialCharTok{+} \FunctionTok{I}\NormalTok{(dfRW}\SpecialCharTok{$}\NormalTok{AGE}\SpecialCharTok{\^{}}\DecValTok{4}\NormalTok{) }\SpecialCharTok{+} \FunctionTok{I}\NormalTok{(dfRW}\SpecialCharTok{$}\NormalTok{AGE}\SpecialCharTok{\^{}}\DecValTok{5}\NormalTok{))}
\FunctionTok{anova}\NormalTok{(f1W,f2W,f3W,f4W,f5W)}
\end{Highlighting}
\end{Shaded}

\begin{verbatim}
## Analysis of Variance Table
## 
## Model 1: dfRW$WM_VOLUME ~ dfRW$AGE + dfRW$SEX
## Model 2: dfRW$WM_VOLUME ~ dfRW$AGE + dfRW$SEX + I(dfRW$AGE^2)
## Model 3: dfRW$WM_VOLUME ~ dfRW$AGE + dfRW$SEX + I(dfRW$AGE^2) + I(dfRW$AGE^3)
## Model 4: dfRW$WM_VOLUME ~ dfRW$AGE + dfRW$SEX + I(dfRW$AGE^2) + I(dfRW$AGE^3) + 
##     I(dfRW$AGE^4)
## Model 5: dfRW$WM_VOLUME ~ dfRW$AGE + dfRW$SEX + I(dfRW$AGE^2) + I(dfRW$AGE^3) + 
##     I(dfRW$AGE^4) + I(dfRW$AGE^5)
##   Res.Df    RSS Df Sum of Sq      F Pr(>F)
## 1    259 474157                           
## 2    258 472150  1    2006.5 1.0955 0.2962
## 3    257 470945  1    1205.6 0.6582 0.4179
## 4    256 467115  1    3829.8 2.0909 0.1494
## 5    255 467060  1      54.7 0.0299 0.8629
\end{verbatim}

NINGUN MODELO CON P VALUE \textless{} 0.05, por lo que tomamos el modelo
1: dfRW\(WM_VOLUME ~ dfRW\)AGE + dfRW\$SEX

\begin{Shaded}
\begin{Highlighting}[]
\FunctionTok{summary}\NormalTok{(f1W) }
\end{Highlighting}
\end{Shaded}

\begin{verbatim}
## 
## Call:
## lm(formula = dfRW$WM_VOLUME ~ dfRW$AGE + dfRW$SEX)
## 
## Residuals:
##      Min       1Q   Median       3Q      Max 
## -118.795  -25.695   -2.713   27.067  169.415 
## 
## Coefficients:
##              Estimate Std. Error t value Pr(>|t|)    
## (Intercept)  355.9619    26.4085  13.479  < 2e-16 ***
## dfRW$AGE       0.9657     0.3505   2.755  0.00629 ** 
## dfRW$SEXMALE  71.7581     5.3702  13.362  < 2e-16 ***
## ---
## Signif. codes:  0 '***' 0.001 '**' 0.01 '*' 0.05 '.' 0.1 ' ' 1
## 
## Residual standard error: 42.79 on 259 degrees of freedom
## Multiple R-squared:  0.4179, Adjusted R-squared:  0.4134 
## F-statistic: 92.96 on 2 and 259 DF,  p-value: < 2.2e-16
\end{verbatim}

p-value: \textless{} 2.2e-16 por lo tanto se acepta el modelo. En el
caso del predictor AGE, si el resto de variables no varían, por cada
unidad de AGE que aumenta el volumen cerebral se aumenta en promedio
0.9657 unidades. Las variables son significativas (al menos `*' 0.05),
Coeficiente de determinacion R\^{}2(0.4179 o 42\%): es el porcentaje de
la variación en la variable de respuesta que es explicado por un modelo
lineal.

\begin{Shaded}
\begin{Highlighting}[]
\FunctionTok{plot.new}\NormalTok{()}
\FunctionTok{plot}\NormalTok{(}\FunctionTok{fitted}\NormalTok{(f1W), }\FunctionTok{residuals}\NormalTok{(f1W), }\AttributeTok{xlab =} \StringTok{"Approached values"}\NormalTok{,  }\AttributeTok{ylab =} \StringTok{"Residuals"}\NormalTok{)}
  \FunctionTok{abline}\NormalTok{(}\AttributeTok{h=}\DecValTok{0}\NormalTok{, }\AttributeTok{lty=}\DecValTok{2}\NormalTok{)}
    \FunctionTok{lines}\NormalTok{(}\FunctionTok{smooth.spline}\NormalTok{(}\FunctionTok{fitted}\NormalTok{(f1W), }\FunctionTok{residuals}\NormalTok{(f1W)))}
\end{Highlighting}
\end{Shaded}

\includegraphics{ProyectoContrastesRegresion_files/figure-latex/unnamed-chunk-51-1.pdf}

\begin{Shaded}
\begin{Highlighting}[]
\FunctionTok{plot}\NormalTok{(f1W}\SpecialCharTok{$}\NormalTok{residuals)}
\end{Highlighting}
\end{Shaded}

\includegraphics{ProyectoContrastesRegresion_files/figure-latex/unnamed-chunk-52-1.pdf}

\begin{Shaded}
\begin{Highlighting}[]
\FunctionTok{summary}\NormalTok{(f1W}\SpecialCharTok{$}\NormalTok{residuals)}
\end{Highlighting}
\end{Shaded}

\begin{verbatim}
##     Min.  1st Qu.   Median     Mean  3rd Qu.     Max. 
## -118.795  -25.695   -2.713    0.000   27.067  169.415
\end{verbatim}

\begin{Shaded}
\begin{Highlighting}[]
\FunctionTok{boxplot}\NormalTok{(f1W}\SpecialCharTok{$}\NormalTok{residuals)}
\end{Highlighting}
\end{Shaded}

\includegraphics{ProyectoContrastesRegresion_files/figure-latex/unnamed-chunk-52-2.pdf}
CARET para modelos del volumen de sustancia blanca.

\begin{Shaded}
\begin{Highlighting}[]
\NormalTok{trainIndex }\OtherTok{\textless{}{-}} \FunctionTok{createDataPartition}\NormalTok{(dfRW}\SpecialCharTok{$}\NormalTok{WM\_VOLUME, }
                                  \AttributeTok{p =}\NormalTok{ .}\DecValTok{8}\NormalTok{, }
                                  \AttributeTok{list =} \ConstantTok{FALSE}\NormalTok{, }
                                  \AttributeTok{times =} \DecValTok{1}\NormalTok{)}

\NormalTok{b\_train }\OtherTok{\textless{}{-}}\NormalTok{ dfRW[trainIndex, ]}
\NormalTok{b\_test  }\OtherTok{\textless{}{-}}\NormalTok{ dfRW[}\SpecialCharTok{{-}}\NormalTok{trainIndex, ]}
\end{Highlighting}
\end{Shaded}

\begin{Shaded}
\begin{Highlighting}[]
\NormalTok{fitControl }\OtherTok{\textless{}{-}} \FunctionTok{trainControl}\NormalTok{(}\AttributeTok{method =} \StringTok{"repeatedcv"}\NormalTok{,}
                           \AttributeTok{number =} \DecValTok{10}\NormalTok{,}
                           \AttributeTok{repeats =} \DecValTok{1}\NormalTok{)}

\CommentTok{\# Y esta función entrena el método}
\NormalTok{caret\_lm\_model }\OtherTok{\textless{}{-}} \FunctionTok{train}\NormalTok{(WM\_VOLUME }\SpecialCharTok{\textasciitilde{}}\NormalTok{ ., }
                        \AttributeTok{data =}\NormalTok{ b\_train, }
                        \AttributeTok{method =} \StringTok{"lm"}\NormalTok{,}
                        \AttributeTok{trControl =}\NormalTok{ fitControl)}
\end{Highlighting}
\end{Shaded}

\begin{Shaded}
\begin{Highlighting}[]
\NormalTok{caret\_lm\_model}
\end{Highlighting}
\end{Shaded}

\begin{verbatim}
## Linear Regression 
## 
## 211 samples
##   2 predictor
## 
## No pre-processing
## Resampling: Cross-Validated (10 fold, repeated 1 times) 
## Summary of sample sizes: 189, 189, 191, 189, 191, 191, ... 
## Resampling results:
## 
##   RMSE      Rsquared   MAE     
##   42.66378  0.4428194  33.44763
## 
## Tuning parameter 'intercept' was held constant at a value of TRUE
\end{verbatim}

Podemos observar que se ha conseguido un MSE = 1825.792.

Comparación entre modelos.

\begin{Shaded}
\begin{Highlighting}[]
\CommentTok{\# Métodos que queremos ejecutar}
\NormalTok{methods }\OtherTok{\textless{}{-}} \FunctionTok{c}\NormalTok{(}\StringTok{"knn"}\NormalTok{, }\StringTok{"gbm"}\NormalTok{, }\StringTok{"lm"}\NormalTok{)}

\CommentTok{\# Los ejecutamos usando lapply}
\NormalTok{methods }\SpecialCharTok{\%\textgreater{}\%} 
  \FunctionTok{lapply}\NormalTok{(}\ControlFlowTok{function}\NormalTok{(x) \{}
    
    \FunctionTok{train}\NormalTok{(WM\_VOLUME }\SpecialCharTok{\textasciitilde{}}\NormalTok{ ., }
          \AttributeTok{data =}\NormalTok{ b\_train,}
          \AttributeTok{method =}\NormalTok{ x,}
          \AttributeTok{trControl =}\NormalTok{ fitControl,}
          \AttributeTok{verbose =} \ConstantTok{FALSE}\NormalTok{)}
    
\NormalTok{  \}) }\OtherTok{{-}\textgreater{}}\NormalTok{ caret\_results}
\end{Highlighting}
\end{Shaded}

\begin{verbatim}
## Warning: In lm.fit(x, y, offset = offset, singular.ok = singular.ok, ...) :
##  extra argument 'verbose' will be disregarded

## Warning: In lm.fit(x, y, offset = offset, singular.ok = singular.ok, ...) :
##  extra argument 'verbose' will be disregarded

## Warning: In lm.fit(x, y, offset = offset, singular.ok = singular.ok, ...) :
##  extra argument 'verbose' will be disregarded

## Warning: In lm.fit(x, y, offset = offset, singular.ok = singular.ok, ...) :
##  extra argument 'verbose' will be disregarded

## Warning: In lm.fit(x, y, offset = offset, singular.ok = singular.ok, ...) :
##  extra argument 'verbose' will be disregarded

## Warning: In lm.fit(x, y, offset = offset, singular.ok = singular.ok, ...) :
##  extra argument 'verbose' will be disregarded

## Warning: In lm.fit(x, y, offset = offset, singular.ok = singular.ok, ...) :
##  extra argument 'verbose' will be disregarded

## Warning: In lm.fit(x, y, offset = offset, singular.ok = singular.ok, ...) :
##  extra argument 'verbose' will be disregarded

## Warning: In lm.fit(x, y, offset = offset, singular.ok = singular.ok, ...) :
##  extra argument 'verbose' will be disregarded

## Warning: In lm.fit(x, y, offset = offset, singular.ok = singular.ok, ...) :
##  extra argument 'verbose' will be disregarded

## Warning: In lm.fit(x, y, offset = offset, singular.ok = singular.ok, ...) :
##  extra argument 'verbose' will be disregarded
\end{verbatim}

\begin{Shaded}
\begin{Highlighting}[]
\CommentTok{\# A cada elemento del resultado le ponemos nombre}
\FunctionTok{names}\NormalTok{(caret\_results) }\OtherTok{\textless{}{-}}\NormalTok{ methods}
\end{Highlighting}
\end{Shaded}

Con esto, somos capaces de seleccionar el mejor de los métodos (el que
minimice el error cometido):

\begin{Shaded}
\begin{Highlighting}[]
\CommentTok{\#install.packages("kableExtra", dependencies = TRUE)}

\CommentTok{\# Recorremos todos los métodos y tomamos el RMSE de cada uno}
\NormalTok{best\_rmse }\OtherTok{\textless{}{-}} \FunctionTok{sapply}\NormalTok{(caret\_results, }\ControlFlowTok{function}\NormalTok{(i) }\FunctionTok{min}\NormalTok{(i}\SpecialCharTok{$}\NormalTok{results}\SpecialCharTok{$}\NormalTok{RMSE, }\AttributeTok{na.rm =} \ConstantTok{TRUE}\NormalTok{))}

\CommentTok{\# El mejor será el que tenga un RMSE más bajo}
\NormalTok{best\_regressor }\OtherTok{\textless{}{-}} \FunctionTok{names}\NormalTok{(best\_rmse)[}\FunctionTok{which.min}\NormalTok{(best\_rmse)]}

\CommentTok{\# Los mostramos en forma de tabla}
\NormalTok{w }\OtherTok{\textless{}{-}}\NormalTok{ best\_rmse }\SpecialCharTok{\%\textgreater{}\%} \FunctionTok{as.data.frame}\NormalTok{()}

\FunctionTok{colnames}\NormalTok{(w) }\OtherTok{\textless{}{-}} \StringTok{"RMSE"}

\NormalTok{w }\SpecialCharTok{\%\textgreater{}\%} 
\NormalTok{  knitr}\SpecialCharTok{::}\FunctionTok{kable}\NormalTok{(}\AttributeTok{format =} \StringTok{"html"}\NormalTok{) }\SpecialCharTok{\%\textgreater{}\%}\NormalTok{ kableExtra}\SpecialCharTok{::}\FunctionTok{kable\_styling}\NormalTok{(}\AttributeTok{font\_size =} \DecValTok{14}\NormalTok{)}
\end{Highlighting}
\end{Shaded}

RMSE

knn

44.19115

gbm

40.63349

lm

42.11834

\begin{Shaded}
\begin{Highlighting}[]
\FunctionTok{cat}\NormalTok{(}\StringTok{"El metodo que minimiza la medida del error:"}\NormalTok{,best\_regressor)}
\end{Highlighting}
\end{Shaded}

\begin{verbatim}
## El metodo que minimiza la medida del error: gbm
\end{verbatim}

Visualización de resultados.

\begin{Shaded}
\begin{Highlighting}[]
\FunctionTok{plot}\NormalTok{(caret\_results}\SpecialCharTok{$}\NormalTok{gbm)}
\end{Highlighting}
\end{Shaded}

\includegraphics{ProyectoContrastesRegresion_files/figure-latex/unnamed-chunk-59-1.pdf}

Usando como predictores las variables de volumen cebrebral de sustancia
gris y blanca, así como el sexo, estimar la edad del individuo en
cuestión utilizando caret.

\begin{Shaded}
\begin{Highlighting}[]
\NormalTok{dfRA}\OtherTok{=}\NormalTok{ df2 }\SpecialCharTok{\%\textgreater{}\%} \FunctionTok{select}\NormalTok{(WM\_VOLUME, GM\_VOLUME,AGE,SEX)}

\NormalTok{trainIndex }\OtherTok{\textless{}{-}} \FunctionTok{createDataPartition}\NormalTok{(dfRA}\SpecialCharTok{$}\NormalTok{AGE, }
                                  \AttributeTok{p =}\NormalTok{ .}\DecValTok{8}\NormalTok{, }
                                  \AttributeTok{list =} \ConstantTok{FALSE}\NormalTok{, }
                                  \AttributeTok{times =} \DecValTok{1}\NormalTok{)}

\NormalTok{b\_train }\OtherTok{\textless{}{-}}\NormalTok{ dfRA[trainIndex, ]}
\NormalTok{b\_test  }\OtherTok{\textless{}{-}}\NormalTok{ dfRA[}\SpecialCharTok{{-}}\NormalTok{trainIndex, ]}
\end{Highlighting}
\end{Shaded}

Comparación entre modelos.

\begin{Shaded}
\begin{Highlighting}[]
\CommentTok{\# Métodos que queremos ejecutar}
\NormalTok{methods }\OtherTok{\textless{}{-}} \FunctionTok{c}\NormalTok{(}\StringTok{"knn"}\NormalTok{, }\StringTok{"gbm"}\NormalTok{, }\StringTok{"lm"}\NormalTok{)}

\CommentTok{\# Los ejecutamos usando lapply}
\NormalTok{methods }\SpecialCharTok{\%\textgreater{}\%} 
  \FunctionTok{lapply}\NormalTok{(}\ControlFlowTok{function}\NormalTok{(x) \{}
    
    \FunctionTok{train}\NormalTok{(AGE }\SpecialCharTok{\textasciitilde{}}\NormalTok{ ., }
          \AttributeTok{data =}\NormalTok{ b\_train,}
          \AttributeTok{method =}\NormalTok{ x,}
          \AttributeTok{trControl =}\NormalTok{ fitControl,}
          \AttributeTok{verbose =} \ConstantTok{FALSE}\NormalTok{)}
    
\NormalTok{  \}) }\OtherTok{{-}\textgreater{}}\NormalTok{ caret\_results}
\end{Highlighting}
\end{Shaded}

\begin{verbatim}
## Warning: In lm.fit(x, y, offset = offset, singular.ok = singular.ok, ...) :
##  extra argument 'verbose' will be disregarded

## Warning: In lm.fit(x, y, offset = offset, singular.ok = singular.ok, ...) :
##  extra argument 'verbose' will be disregarded

## Warning: In lm.fit(x, y, offset = offset, singular.ok = singular.ok, ...) :
##  extra argument 'verbose' will be disregarded

## Warning: In lm.fit(x, y, offset = offset, singular.ok = singular.ok, ...) :
##  extra argument 'verbose' will be disregarded

## Warning: In lm.fit(x, y, offset = offset, singular.ok = singular.ok, ...) :
##  extra argument 'verbose' will be disregarded

## Warning: In lm.fit(x, y, offset = offset, singular.ok = singular.ok, ...) :
##  extra argument 'verbose' will be disregarded

## Warning: In lm.fit(x, y, offset = offset, singular.ok = singular.ok, ...) :
##  extra argument 'verbose' will be disregarded

## Warning: In lm.fit(x, y, offset = offset, singular.ok = singular.ok, ...) :
##  extra argument 'verbose' will be disregarded

## Warning: In lm.fit(x, y, offset = offset, singular.ok = singular.ok, ...) :
##  extra argument 'verbose' will be disregarded

## Warning: In lm.fit(x, y, offset = offset, singular.ok = singular.ok, ...) :
##  extra argument 'verbose' will be disregarded

## Warning: In lm.fit(x, y, offset = offset, singular.ok = singular.ok, ...) :
##  extra argument 'verbose' will be disregarded
\end{verbatim}

\begin{Shaded}
\begin{Highlighting}[]
\CommentTok{\# A cada elemento del resultado le ponemos nombre}
\FunctionTok{names}\NormalTok{(caret\_results) }\OtherTok{\textless{}{-}}\NormalTok{ methods}
\end{Highlighting}
\end{Shaded}

Con esto, somos capaces de seleccionar el mejor de los métodos (el que
minimice el error cometido):

\begin{Shaded}
\begin{Highlighting}[]
\CommentTok{\#install.packages("kableExtra", dependencies = TRUE)}

\CommentTok{\# Recorremos todos los métodos y tomamos el RMSE de cada uno}
\NormalTok{best\_rmse }\OtherTok{\textless{}{-}} \FunctionTok{sapply}\NormalTok{(caret\_results, }\ControlFlowTok{function}\NormalTok{(i) }\FunctionTok{min}\NormalTok{(i}\SpecialCharTok{$}\NormalTok{results}\SpecialCharTok{$}\NormalTok{RMSE, }\AttributeTok{na.rm =} \ConstantTok{TRUE}\NormalTok{))}

\CommentTok{\# El mejor será el que tenga un RMSE más bajo}
\NormalTok{best\_regressor }\OtherTok{\textless{}{-}} \FunctionTok{names}\NormalTok{(best\_rmse)[}\FunctionTok{which.min}\NormalTok{(best\_rmse)]}

\CommentTok{\# Los mostramos en forma de tabla}
\NormalTok{w }\OtherTok{\textless{}{-}}\NormalTok{ best\_rmse }\SpecialCharTok{\%\textgreater{}\%} \FunctionTok{as.data.frame}\NormalTok{()}

\FunctionTok{colnames}\NormalTok{(w) }\OtherTok{\textless{}{-}} \StringTok{"RMSE"}

\NormalTok{w }\SpecialCharTok{\%\textgreater{}\%} 
\NormalTok{  knitr}\SpecialCharTok{::}\FunctionTok{kable}\NormalTok{(}\AttributeTok{format =} \StringTok{"html"}\NormalTok{) }\SpecialCharTok{\%\textgreater{}\%}\NormalTok{ kableExtra}\SpecialCharTok{::}\FunctionTok{kable\_styling}\NormalTok{(}\AttributeTok{font\_size =} \DecValTok{14}\NormalTok{)}
\end{Highlighting}
\end{Shaded}

RMSE

knn

7.179391

gbm

6.780608

lm

7.052856

\begin{Shaded}
\begin{Highlighting}[]
\FunctionTok{cat}\NormalTok{(}\StringTok{"El metodo que minimiza la medida del error:"}\NormalTok{,best\_regressor)}
\end{Highlighting}
\end{Shaded}

\begin{verbatim}
## El metodo que minimiza la medida del error: gbm
\end{verbatim}

\begin{Shaded}
\begin{Highlighting}[]
\NormalTok{fitControl }\OtherTok{\textless{}{-}} \FunctionTok{trainControl}\NormalTok{(}\AttributeTok{method =} \StringTok{"repeatedcv"}\NormalTok{,}
                           \AttributeTok{number =} \DecValTok{10}\NormalTok{,}
                           \AttributeTok{repeats =} \DecValTok{1}\NormalTok{)}

\CommentTok{\# Y esta función entrena el método}
\NormalTok{caret\_lm\_model }\OtherTok{\textless{}{-}} \FunctionTok{train}\NormalTok{(AGE }\SpecialCharTok{\textasciitilde{}}\NormalTok{ ., }
                        \AttributeTok{data =}\NormalTok{ b\_train, }
                        \AttributeTok{method =} \StringTok{"lm"}\NormalTok{,}
                        \AttributeTok{trControl =}\NormalTok{ fitControl)}
\end{Highlighting}
\end{Shaded}

Visualización de resultados.

\begin{Shaded}
\begin{Highlighting}[]
\NormalTok{caret\_results}\SpecialCharTok{$}\NormalTok{lm}
\end{Highlighting}
\end{Shaded}

\begin{verbatim}
## Linear Regression 
## 
## 211 samples
##   3 predictor
## 
## No pre-processing
## Resampling: Cross-Validated (10 fold, repeated 1 times) 
## Summary of sample sizes: 189, 189, 190, 191, 191, 190, ... 
## Resampling results:
## 
##   RMSE      Rsquared   MAE     
##   7.052856  0.1695028  5.463117
## 
## Tuning parameter 'intercept' was held constant at a value of TRUE
\end{verbatim}

\#Regresión logística y clasificación Con la regresión logística podemos
asumir valores para variables binarias.

Queremos poder decidir, a partir de los datos de volumetría cerebral
contemplados, si un nuevo sujeto puede padecer la enfermedad de
Alzheimer.

Para ello, construiremos modelos de regresión logística y de
clasificación para resolver ese problema.

Debemos dividir el dataset en un conjunto de entrenamiento con el 80\%
de los datos, seleccionado aleatoriamente, y un 20\% restante para
validar los modelos.

\begin{Shaded}
\begin{Highlighting}[]
\NormalTok{dfRL}\OtherTok{=}\NormalTok{ df2 }\SpecialCharTok{\%\textgreater{}\%} \FunctionTok{select}\NormalTok{(BRAIN\_VOLUME,WM\_VOLUME, GM\_VOLUME,AGE,SEX, CLASS)}
\FunctionTok{set.seed}\NormalTok{(}\DecValTok{123}\NormalTok{)}

\NormalTok{dfRL}\SpecialCharTok{$}\NormalTok{SEX }\OtherTok{=} \FunctionTok{as.factor}\NormalTok{(dfRL}\SpecialCharTok{$}\NormalTok{SEX)}
\NormalTok{dfRL}\SpecialCharTok{$}\NormalTok{CLASS }\OtherTok{=} \FunctionTok{as.factor}\NormalTok{(dfRL}\SpecialCharTok{$}\NormalTok{CLASS)}
\NormalTok{trainIndex }\OtherTok{\textless{}{-}} \FunctionTok{createDataPartition}\NormalTok{(dfRL}\SpecialCharTok{$}\NormalTok{CLASS, }
                                  \AttributeTok{p =}\NormalTok{ .}\DecValTok{8}\NormalTok{, }
                                  \AttributeTok{list =} \ConstantTok{FALSE}\NormalTok{, }
                                  \AttributeTok{times =} \DecValTok{1}\NormalTok{)}

\NormalTok{b\_train }\OtherTok{\textless{}{-}}\NormalTok{ dfRL[trainIndex, ]}
\NormalTok{b\_test  }\OtherTok{\textless{}{-}}\NormalTok{ dfRL[}\SpecialCharTok{{-}}\NormalTok{trainIndex, ]}
\end{Highlighting}
\end{Shaded}

\begin{Shaded}
\begin{Highlighting}[]
\NormalTok{gm }\OtherTok{\textless{}{-}} \FunctionTok{glm}\NormalTok{(}\AttributeTok{formula =}\NormalTok{ CLASS }\SpecialCharTok{\textasciitilde{}}\NormalTok{ BRAIN\_VOLUME  }\SpecialCharTok{+}\NormalTok{ WM\_VOLUME }\SpecialCharTok{+}\NormalTok{ GM\_VOLUME }\SpecialCharTok{+}\NormalTok{ AGE }\SpecialCharTok{+}\NormalTok{ SEX, }\AttributeTok{data =}\NormalTok{ b\_train, }\AttributeTok{family =}\NormalTok{ binomial)}
\FunctionTok{summary}\NormalTok{(gm)}
\end{Highlighting}
\end{Shaded}

\begin{verbatim}
## 
## Call:
## glm(formula = CLASS ~ BRAIN_VOLUME + WM_VOLUME + GM_VOLUME + 
##     AGE + SEX, family = binomial, data = b_train)
## 
## Deviance Residuals: 
##     Min       1Q   Median       3Q      Max  
## -2.7349   0.2763   0.3918   0.5279   1.1665  
## 
## Coefficients:
##               Estimate Std. Error z value Pr(>|z|)    
## (Intercept)  -0.755961   3.378187  -0.224 0.822931    
## BRAIN_VOLUME -0.022006   0.007108  -3.096 0.001963 ** 
## WM_VOLUME     0.027491   0.011415   2.408 0.016025 *  
## GM_VOLUME     0.030768   0.008819   3.489 0.000485 ***
## AGE           0.031391   0.031152   1.008 0.313606    
## SEXMALE       0.501730   0.660292   0.760 0.447337    
## ---
## Signif. codes:  0 '***' 0.001 '**' 0.01 '*' 0.05 '.' 0.1 ' ' 1
## 
## (Dispersion parameter for binomial family taken to be 1)
## 
##     Null deviance: 149.26  on 209  degrees of freedom
## Residual deviance: 133.74  on 204  degrees of freedom
## AIC: 145.74
## 
## Number of Fisher Scoring iterations: 5
\end{verbatim}

EL modelo nos indica que los predictores volumen cerebral, sustancia
blanca y gris son signiicativos, y que los predictores AGE y SEX pueden
ser 0 por lo que serán omitidos para evitar ruido y mermar complejidad
al modelo.

\begin{Shaded}
\begin{Highlighting}[]
\NormalTok{gm }\OtherTok{\textless{}{-}} \FunctionTok{glm}\NormalTok{(}\AttributeTok{formula =}\NormalTok{ CLASS }\SpecialCharTok{\textasciitilde{}}\NormalTok{ BRAIN\_VOLUME  }\SpecialCharTok{+}\NormalTok{ WM\_VOLUME }\SpecialCharTok{+}\NormalTok{ GM\_VOLUME, }\AttributeTok{data =}\NormalTok{ b\_train, }\AttributeTok{family =}\NormalTok{ binomial)}
\FunctionTok{summary}\NormalTok{(gm)}
\end{Highlighting}
\end{Shaded}

\begin{verbatim}
## 
## Call:
## glm(formula = CLASS ~ BRAIN_VOLUME + WM_VOLUME + GM_VOLUME, family = binomial, 
##     data = b_train)
## 
## Deviance Residuals: 
##     Min       1Q   Median       3Q      Max  
## -2.7312   0.2638   0.3956   0.5485   1.0208  
## 
## Coefficients:
##               Estimate Std. Error z value Pr(>|z|)    
## (Intercept)   0.159136   2.110416   0.075 0.939892    
## BRAIN_VOLUME -0.019525   0.006652  -2.935 0.003333 ** 
## WM_VOLUME     0.026763   0.011241   2.381 0.017270 *  
## GM_VOLUME     0.028264   0.008479   3.333 0.000858 ***
## ---
## Signif. codes:  0 '***' 0.001 '**' 0.01 '*' 0.05 '.' 0.1 ' ' 1
## 
## (Dispersion parameter for binomial family taken to be 1)
## 
##     Null deviance: 149.26  on 209  degrees of freedom
## Residual deviance: 135.16  on 206  degrees of freedom
## AIC: 143.16
## 
## Number of Fisher Scoring iterations: 5
\end{verbatim}

probar precison sobre el conjunto de validación

\begin{Shaded}
\begin{Highlighting}[]
\NormalTok{logit }\OtherTok{\textless{}{-}} \FunctionTok{predict}\NormalTok{(gm,b\_test)}
\NormalTok{p }\OtherTok{\textless{}{-}} \DecValTok{1} \SpecialCharTok{/}\NormalTok{ (}\DecValTok{1} \SpecialCharTok{+} \FunctionTok{exp}\NormalTok{(}\SpecialCharTok{{-}}\NormalTok{logit))}
\NormalTok{p}
\end{Highlighting}
\end{Shaded}

\begin{verbatim}
##         1         2         3         4         5         6         7         8 
## 0.7274061 0.3787791 0.8467967 0.7318253 0.8275232 0.9781014 0.9130623 0.7390191 
##         9        10        11        12        13        14        15        16 
## 0.9578022 0.7478369 0.8323747 0.9792444 0.9512721 0.9492280 0.6901570 0.9586999 
##        17        18        19        20        21        22        23        24 
## 0.7333317 0.9461054 0.9688220 0.9343576 0.5164343 0.9003090 0.8785173 0.3836110 
##        25        26        27        28        29        30        31        32 
## 0.9688220 0.8738102 0.8436649 0.8979713 0.7377519 0.8670524 0.8955731 0.9476784 
##        33        34        35        36        37        38        39        40 
## 0.9032050 0.9714633 0.9536468 0.8756360 0.8016609 0.9625171 0.9489705 0.9609344 
##        41        42        43        44        45        46        47        48 
## 0.8655487 0.7240897 0.8580354 0.8395252 0.8381907 0.8878290 0.8852617 0.8090749 
##        49        50        51        52 
## 0.9837312 0.9294216 0.7714408 0.9331627
\end{verbatim}

Crear un modelo de regresión logística adecuado (especialmente, en el
que los predictores sean significativos) para la variable CLASS. ¿Qué
precisión tiene ese modelo si lo ejecutamos (usando predict()) sobre el
conjunto de validación?

Asimismo, se desea construir una serie de modelos de clasificación
binaria con caret que, utilizando todos los parámetros disponibles,
estime la clase (CLASS) a la que pertenece cada individuo. Para ello:

\begin{Shaded}
\begin{Highlighting}[]
\NormalTok{folds }\OtherTok{\textless{}{-}} \FunctionTok{createFolds}\NormalTok{(dfRL}\SpecialCharTok{$}\NormalTok{CLASS, }\AttributeTok{k =} \DecValTok{5}\NormalTok{)}
\CommentTok{\# Validación cruzada}
\FunctionTok{set.seed}\NormalTok{(}\DecValTok{123}\NormalTok{)}
\NormalTok{fitControl }\OtherTok{\textless{}{-}} \FunctionTok{trainControl}\NormalTok{(}\AttributeTok{method =} \StringTok{"repeatedcv"}\NormalTok{,}
                           \AttributeTok{number =} \DecValTok{5}\NormalTok{,}
                           \AttributeTok{repeats =} \DecValTok{1}\NormalTok{)}
\end{Highlighting}
\end{Shaded}

\begin{Shaded}
\begin{Highlighting}[]
\NormalTok{methods }\OtherTok{\textless{}{-}} \FunctionTok{c}\NormalTok{(}\StringTok{"svmLinear"}\NormalTok{, }\StringTok{"rpart"}\NormalTok{, }\StringTok{"regLogistic"}\NormalTok{, }\StringTok{"C5.0"}\NormalTok{)}

\CommentTok{\# Se ejecutan}
\NormalTok{methods }\SpecialCharTok{\%\textgreater{}\%} 
  \FunctionTok{lapply}\NormalTok{(}\ControlFlowTok{function}\NormalTok{(x) \{}
    
\NormalTok{    caret}\SpecialCharTok{::}\FunctionTok{train}\NormalTok{(CLASS }\SpecialCharTok{\textasciitilde{}}\NormalTok{ ., }
                 \AttributeTok{data =}\NormalTok{ b\_train,}
                 \AttributeTok{method =}\NormalTok{ x,}
                 \AttributeTok{trControl =}\NormalTok{ fitControl)}
    
\NormalTok{  \}) }\OtherTok{{-}\textgreater{}}\NormalTok{ caret\_class\_results}
\end{Highlighting}
\end{Shaded}

\begin{verbatim}
## Warning: 'trials' should be <= 1 for this object. Predictions generated using 1
## trials

## Warning: 'trials' should be <= 1 for this object. Predictions generated using 1
## trials

## Warning: 'trials' should be <= 1 for this object. Predictions generated using 1
## trials

## Warning: 'trials' should be <= 1 for this object. Predictions generated using 1
## trials

## Warning: 'trials' should be <= 1 for this object. Predictions generated using 1
## trials

## Warning: 'trials' should be <= 1 for this object. Predictions generated using 1
## trials

## Warning: 'trials' should be <= 1 for this object. Predictions generated using 1
## trials

## Warning: 'trials' should be <= 1 for this object. Predictions generated using 1
## trials

## Warning: 'trials' should be <= 1 for this object. Predictions generated using 1
## trials

## Warning: 'trials' should be <= 1 for this object. Predictions generated using 1
## trials

## Warning: 'trials' should be <= 1 for this object. Predictions generated using 1
## trials

## Warning: 'trials' should be <= 1 for this object. Predictions generated using 1
## trials

## Warning: 'trials' should be <= 1 for this object. Predictions generated using 1
## trials

## Warning: 'trials' should be <= 1 for this object. Predictions generated using 1
## trials

## Warning: 'trials' should be <= 1 for this object. Predictions generated using 1
## trials

## Warning: 'trials' should be <= 1 for this object. Predictions generated using 1
## trials

## Warning: 'trials' should be <= 1 for this object. Predictions generated using 1
## trials

## Warning: 'trials' should be <= 1 for this object. Predictions generated using 1
## trials

## Warning: 'trials' should be <= 1 for this object. Predictions generated using 1
## trials

## Warning: 'trials' should be <= 1 for this object. Predictions generated using 1
## trials
\end{verbatim}

\begin{Shaded}
\begin{Highlighting}[]
\FunctionTok{names}\NormalTok{(caret\_class\_results) }\OtherTok{\textless{}{-}}\NormalTok{ methods}
\end{Highlighting}
\end{Shaded}

\begin{Shaded}
\begin{Highlighting}[]
\CommentTok{\# Recorremos todos los resultados, almacenando la métrica}
\CommentTok{\# accuracy (la precisión)}
\NormalTok{best\_acc }\OtherTok{\textless{}{-}} \FunctionTok{sapply}\NormalTok{(caret\_class\_results, }\ControlFlowTok{function}\NormalTok{(i) }\FunctionTok{max}\NormalTok{(i}\SpecialCharTok{$}\NormalTok{results}\SpecialCharTok{$}\NormalTok{Accuracy, }\AttributeTok{na.rm =} \ConstantTok{TRUE}\NormalTok{))}

\NormalTok{best\_classifier }\OtherTok{\textless{}{-}} \FunctionTok{names}\NormalTok{(best\_acc)[}\FunctionTok{which.max}\NormalTok{(best\_acc)]}

\NormalTok{w }\OtherTok{\textless{}{-}}\NormalTok{ best\_acc }\SpecialCharTok{\%\textgreater{}\%} \FunctionTok{as.data.frame}\NormalTok{()}

\FunctionTok{colnames}\NormalTok{(w) }\OtherTok{\textless{}{-}} \StringTok{"Accuracy"}

\NormalTok{w }\SpecialCharTok{\%\textgreater{}\%} 
\NormalTok{  knitr}\SpecialCharTok{::}\FunctionTok{kable}\NormalTok{(}\AttributeTok{format =} \StringTok{"html"}\NormalTok{) }\SpecialCharTok{\%\textgreater{}\%} 
\NormalTok{  kableExtra}\SpecialCharTok{::}\FunctionTok{kable\_styling}\NormalTok{(}\AttributeTok{font\_size =} \DecValTok{14}\NormalTok{)}
\end{Highlighting}
\end{Shaded}

Accuracy

svmLinear

0.8858034

rpart

0.8858034

regLogistic

0.8905653

C5.0

0.8858034

En este caso, el mejor clasificador ha sido regLogistic con una
precisión de 89\%.

\begin{Shaded}
\begin{Highlighting}[]
\NormalTok{predicted\_class }\OtherTok{\textless{}{-}}\NormalTok{ caret\_class\_results[[best\_classifier]] }\SpecialCharTok{\%\textgreater{}\%}  
  \FunctionTok{predict}\NormalTok{(b\_test)}

\FunctionTok{confusionMatrix}\NormalTok{(predicted\_class, b\_test}\SpecialCharTok{$}\NormalTok{CLASS)}
\end{Highlighting}
\end{Shaded}

\begin{verbatim}
## Confusion Matrix and Statistics
## 
##           Reference
## Prediction AD HEALTHY
##    AD       1       1
##    HEALTHY  5      45
##                                           
##                Accuracy : 0.8846          
##                  95% CI : (0.7656, 0.9565)
##     No Information Rate : 0.8846          
##     P-Value [Acc > NIR] : 0.6065          
##                                           
##                   Kappa : 0.2041          
##                                           
##  Mcnemar's Test P-Value : 0.2207          
##                                           
##             Sensitivity : 0.16667         
##             Specificity : 0.97826         
##          Pos Pred Value : 0.50000         
##          Neg Pred Value : 0.90000         
##              Prevalence : 0.11538         
##          Detection Rate : 0.01923         
##    Detection Prevalence : 0.03846         
##       Balanced Accuracy : 0.57246         
##                                           
##        'Positive' Class : AD              
## 
\end{verbatim}

\begin{Shaded}
\begin{Highlighting}[]
\FunctionTok{plot}\NormalTok{(caret\_class\_results[[best\_classifier]])}
\end{Highlighting}
\end{Shaded}

\includegraphics{ProyectoContrastesRegresion_files/figure-latex/unnamed-chunk-74-1.pdf}

\# Inferencia Bayesiana

Estudio de la incidencia de la enfermedad de Alzheimer en la
población.estima que, a priori, la incidencia de la enfermedad de
Alzheimer en la población mayor de 60 años está entre un 7\% y un 9\%.

Para comprobarlo, usaremos inferencia bayesiana con el dataset que
estamos estudiando:

Estimar la distribución a posteriori de la proporción θ de casos de
Alzheimer, partiendo de un prior no informativo (uniforme en {[}0,1{]}),
y partiendo de un prior con forma de trapecio, tal y como se ve en la
siguiente figura:

\begin{Shaded}
\begin{Highlighting}[]
\CommentTok{\#theta =c(0.07,0.08,0.09)}
\CommentTok{\#pTheta \textless{}{-} pmin(theta, 1 {-} theta)}
\CommentTok{\#pTheta}
\CommentTok{\#pDatosDadoTheta \textless{}{-} choose(N, nCaras) * theta\^{} nCaras * (1 {-} theta)\^{}nCruz}
\CommentTok{\#plot(theta, pTheta)}
\end{Highlighting}
\end{Shaded}

\#\#Bibliografias

\url{https://blog.minitab.com/es/analisis-de-regresion-como-puedo-interpretar-el-r-cuadrado-y-evaluar-la-bondad-de-ajuste}

\end{document}
